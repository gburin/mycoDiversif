\documentclass[]{article}
\usepackage{lmodern}
\usepackage{amssymb,amsmath}
\usepackage{ifxetex,ifluatex}
\usepackage{fixltx2e} % provides \textsubscript
\ifnum 0\ifxetex 1\fi\ifluatex 1\fi=0 % if pdftex
  \usepackage[T1]{fontenc}
  \usepackage[utf8]{inputenc}
\else % if luatex or xelatex
  \ifxetex
    \usepackage{mathspec}
  \else
    \usepackage{fontspec}
  \fi
  \defaultfontfeatures{Ligatures=TeX,Scale=MatchLowercase}
\fi
% use upquote if available, for straight quotes in verbatim environments
\IfFileExists{upquote.sty}{\usepackage{upquote}}{}
% use microtype if available
\IfFileExists{microtype.sty}{%
\usepackage{microtype}
\UseMicrotypeSet[protrusion]{basicmath} % disable protrusion for tt fonts
}{}
\usepackage[margin=1in]{geometry}
\usepackage{hyperref}
\hypersetup{unicode=true,
            pdftitle={Supplementary Material - Mujica et al.},
            pdfborder={0 0 0},
            breaklinks=true}
\urlstyle{same}  % don't use monospace font for urls
\usepackage{graphicx,grffile}
\makeatletter
\def\maxwidth{\ifdim\Gin@nat@width>\linewidth\linewidth\else\Gin@nat@width\fi}
\def\maxheight{\ifdim\Gin@nat@height>\textheight\textheight\else\Gin@nat@height\fi}
\makeatother
% Scale images if necessary, so that they will not overflow the page
% margins by default, and it is still possible to overwrite the defaults
% using explicit options in \includegraphics[width, height, ...]{}
\setkeys{Gin}{width=\maxwidth,height=\maxheight,keepaspectratio}
\IfFileExists{parskip.sty}{%
\usepackage{parskip}
}{% else
\setlength{\parindent}{0pt}
\setlength{\parskip}{6pt plus 2pt minus 1pt}
}
\setlength{\emergencystretch}{3em}  % prevent overfull lines
\providecommand{\tightlist}{%
  \setlength{\itemsep}{0pt}\setlength{\parskip}{0pt}}
\setcounter{secnumdepth}{5}
% Redefines (sub)paragraphs to behave more like sections
\ifx\paragraph\undefined\else
\let\oldparagraph\paragraph
\renewcommand{\paragraph}[1]{\oldparagraph{#1}\mbox{}}
\fi
\ifx\subparagraph\undefined\else
\let\oldsubparagraph\subparagraph
\renewcommand{\subparagraph}[1]{\oldsubparagraph{#1}\mbox{}}
\fi

%%% Use protect on footnotes to avoid problems with footnotes in titles
\let\rmarkdownfootnote\footnote%
\def\footnote{\protect\rmarkdownfootnote}

%%% Change title format to be more compact
\usepackage{titling}

% Create subtitle command for use in maketitle
\providecommand{\subtitle}[1]{
  \posttitle{
    \begin{center}\large#1\end{center}
    }
}

\setlength{\droptitle}{-2em}

  \title{Supplementary Material - Mujica et al.}
    \pretitle{\vspace{\droptitle}\centering\huge}
  \posttitle{\par}
    \author{}
    \preauthor{}\postauthor{}
    \date{}
    \predate{}\postdate{}
  
\usepackage{float}
\floatplacement{figure}{H}
\floatplacement{table}{H}
\floatplacement{verbatim}{H}

\begin{document}
\maketitle

\hypertarget{phylogenetic-signal-of-mycorrhizal-types}{%
\section{Phylogenetic signal of Mycorrhizal
types}\label{phylogenetic-signal-of-mycorrhizal-types}}

\begin{table}[H]

\caption{\label{tab:unnamed-chunk-2}P-values for test of phylogenetic signal (D) of each mycorrhizal type. Significant values are highlighted in bold.}
\centering
\begin{tabular}{r|l|l|l|l|l}
\hline
Threshold & model & AM & EM & MIX & NM\\
\hline
50 & random & \textbf{0} & \textbf{0} & 0.842 & \textbf{0}\\
\hline
50 & BM & 0.899 & 0.969 & \textbf{0.043} & 0.832\\
\hline
60 & random & \textbf{0} & \textbf{0} & \textbf{0.031} & \textbf{0}\\
\hline
60 & BM & 0.88 & 0.949 & 0.215 & 0.852\\
\hline
80 & random & \textbf{0} & \textbf{0} & \textbf{0} & \textbf{0}\\
\hline
80 & BM & 0.821 & 0.956 & 0.228 & 0.747\\
\hline
100 & random & \textbf{0} & \textbf{0} & 0.247 & \textbf{0}\\
\hline
100 & BM & 0.083 & 0.958 & \textbf{0} & 0.643\\
\hline
\end{tabular}
\end{table}

\hypertarget{analysis-per-genera-excluding-100-mix-families}{%
\section{Analysis per genera excluding 100\% MIX
families}\label{analysis-per-genera-excluding-100-mix-families}}

\hypertarget{boxplots}{%
\subsection{Boxplots}\label{boxplots}}

\hypertarget{threshold-50}\label{threshold-50}}

\begin{figure}
\centering
\includegraphics{supp_mat_files/figure-latex/unnamed-chunk-3-1.pdf}
\caption{Relationship between mycorrhizal type and diversification
rates. a) diversification rate estimated with epsilon = 0 and b)
diversification rate estimated with epsilon = 0.9. AM: Arbuscular
mycorrhiza, EM: Ectomycorrhiza, NM: non-mycorrhizal and MIX (families
with no dominance of any specific mycorrhizal association). The size of
the points indicates the Mycorrhizal Type Diversity Index value for each
lineage, indicating a predominance of larger indices with higher
diversification rates.}
\end{figure}

\hypertarget{threshold-60}\label{threshold-60}}

\begin{figure}
\centering
\includegraphics{supp_mat_files/figure-latex/unnamed-chunk-4-1.pdf}
\caption{Relationship between mycorrhizal type and diversification
rates. a) diversification rate estimated with epsilon = 0 and b)
diversification rate estimated with epsilon = 0.9. AM: Arbuscular
mycorrhiza, EM: Ectomycorrhiza, NM: non-mycorrhizal and MIX (families
with no dominance of any specific mycorrhizal association). The size of
the points indicates the Mycorrhizal Type Diversity Index value for each
lineage, indicating a predominance of larger indices with higher
diversification rates.}
\end{figure}

\hypertarget{threshold-80}\label{threshold-80}}

\begin{figure}
\centering
\includegraphics{supp_mat_files/figure-latex/unnamed-chunk-5-1.pdf}
\caption{Relationship between mycorrhizal type and diversification
rates. a) diversification rate estimated with epsilon = 0 and b)
diversification rate estimated with epsilon = 0.9. AM: Arbuscular
mycorrhiza, EM: Ectomycorrhiza, NM: non-mycorrhizal and MIX (families
with no dominance of any specific mycorrhizal association). The size of
the points indicates the Mycorrhizal Type Diversity Index value for each
lineage, indicating a predominance of larger indices with higher
diversification rates.}
\end{figure}

\hypertarget{threshold-100}\label{threshold-100}}

\begin{figure}
\centering
\includegraphics{supp_mat_files/figure-latex/unnamed-chunk-6-1.pdf}
\caption{Relationship between mycorrhizal type and diversification
rates. a) diversification rate estimated with epsilon = 0 and b)
diversification rate estimated with epsilon = 0.9. AM: Arbuscular
mycorrhiza, EM: Ectomycorrhiza, NM: non-mycorrhizal and MIX (families
with no dominance of any specific mycorrhizal association). The size of
the points indicates the Mycorrhizal Type Diversity Index value for each
lineage, indicating a predominance of larger indices with higher
diversification rates.}
\end{figure}

\pagebreak

\hypertarget{summary-statistics}{%
\subsection{Summary statistics}\label{summary-statistics}}

\hypertarget{phyanova}{%
\subsubsection{phyANOVA}\label{phyanova}}

\begin{table}[H]

\caption{\label{tab:unnamed-chunk-7}Summary statistics for phyANOVA for both values of epsilon. Significant values are highlighted in bold.}
\centering
\begin{tabular}{r|l|l|l|l}
\hline
Threshold & F.r0 & pvalue.r0 & F.r09 & pvalue.r09\\
\hline
50 & 3.996 & 0.089 & 3.147 & 0.162\\
\hline
60 & 7.255 & \textbf{0.013} & 7.35 & \textbf{0.007}\\
\hline
80 & 14.181 & \textbf{0.001} & 13.177 & \textbf{0.001}\\
\hline
100 & 44.007 & \textbf{0.001} & 52.476 & \textbf{0.001}\\
\hline
\end{tabular}
\end{table}

\hypertarget{standard-anova}{%
\subsubsection{Standard ANOVA}\label{standard-anova}}

\begin{table}[H]

\caption{\label{tab:unnamed-chunk-8}summary statistics for standard ANOVA for both values of epsilon. Significant values are highlighted in bold.}
\centering
\begin{tabular}{r|l|l|l|l}
\hline
Threshold & F.r0 & pvalue.r0 & F.r09 & pvalue.r09\\
\hline
50 & 3.996 & \textbf{0.008} & 3.147 & \textbf{0.025}\\
\hline
60 & 7.255 & \textbf{0} & 7.35 & \textbf{0}\\
\hline
80 & 14.181 & \textbf{0} & 13.177 & \textbf{0}\\
\hline
100 & 44.007 & \textbf{0} & 52.476 & \textbf{0}\\
\hline
\end{tabular}
\end{table}

\hypertarget{posthoc-tests}{%
\subsection{Posthoc tests}\label{posthoc-tests}}

\hypertarget{phyanova-1}{%
\subsubsection{phyANOVA}\label{phyanova-1}}

\begin{table}[H]

\caption{\label{tab:unnamed-chunk-9}Pairwise Corrected p-values for phyANOVA. Significant values are highlighted in bold.}
\centering
\begin{tabular}{r|l|l|l|l|l|l|l|l|l}
\hline
Threshold & Mycorrhizal.Type & AM.r0 & EM.r0 & MIX.r0 & NM.r0 & AM.r09 & EM.r09 & MIX.r09 & NM.r09\\
\hline
50 & AM & 1 & 0.858 & \textbf{0.042} & 0.747 & 1 & 1 & \textbf{0.048} & 1\\
\hline
50 & EM & 0.858 & 1 & 0.716 & 0.858 & 1 & 1 & 0.62 & 1\\
\hline
50 & MIX & \textbf{0.042} & 0.716 & 1 & \textbf{0.042} & \textbf{0.048} & 0.62 & 1 & \textbf{0.048}\\
\hline
50 & NM & 0.747 & 0.858 & \textbf{0.042} & 1 & 1 & 1 & \textbf{0.048} & 1\\
\hline
60 & AM & 1 & 1 & \textbf{0.006} & 1 & 1 & 1 & \textbf{0.006} & 1\\
\hline
60 & EM & 1 & 1 & 1 & 1 & 1 & 1 & 0.952 & 1\\
\hline
60 & MIX & \textbf{0.006} & 1 & 1 & \textbf{0.006} & \textbf{0.006} & 0.952 & 1 & \textbf{0.006}\\
\hline
60 & NM & 1 & 1 & \textbf{0.006} & 1 & 1 & 1 & \textbf{0.006} & 1\\
\hline
80 & AM & 1 & 1 & \textbf{0.006} & 1 & 1 & 1 & \textbf{0.006} & 1\\
\hline
80 & EM & 1 & 1 & 1 & 1 & 1 & 1 & 1 & 1\\
\hline
80 & MIX & \textbf{0.006} & 1 & 1 & \textbf{0.006} & \textbf{0.006} & 1 & 1 & \textbf{0.006}\\
\hline
80 & NM & 1 & 1 & \textbf{0.006} & 1 & 1 & 1 & \textbf{0.006} & 1\\
\hline
100 & AM & 1 & 0.552 & \textbf{0.006} & 0.868 & 1 & 0.567 & \textbf{0.006} & 0.627\\
\hline
100 & EM & 0.552 & 1 & 0.552 & 0.552 & 0.567 & 1 & 0.468 & 0.567\\
\hline
100 & MIX & \textbf{0.006} & 0.552 & 1 & \textbf{0.006} & \textbf{0.006} & 0.468 & 1 & \textbf{0.006}\\
\hline
100 & NM & 0.868 & 0.552 & \textbf{0.006} & 1 & 0.627 & 0.567 & \textbf{0.006} & 1\\
\hline
\end{tabular}
\end{table}

\hypertarget{standard-anova-1}{%
\subsubsection{Standard ANOVA}\label{standard-anova-1}}

\begin{table}[H]

\caption{\label{tab:unnamed-chunk-10}Pairwise Corrected p-values for standard ANOVA. Significant values are highlighted in bold.}
\centering
\begin{tabular}{l|l|l|l|l|l|l|l|l}
\hline
Types & 50.r0 & 60.r0 & 80.r0 & 100.r0 & 50.r09 & 60.r09 & 80.r09 & 100.r09\\
\hline
EM-AM & 0.6181 & 6.86e-01 & 5.80e-01 & 1.54e-01 & 0.8536 & 8.66e-01 & 7.94e-01 & 0.2459\\
\hline
MIX-AM & \textbf{0.0175} & 4.19e-05 & 8.91e-09 & \textbf{0.00e+00} & \textbf{0.0253} & 4.66e-05 & 7.23e-08 & \textbf{0.0000}\\
\hline
NM-AM & 0.3644 & 1.00e+00 & 7.83e-01 & 9.97e-01 & 0.6635 & 9.40e-01 & 4.56e-01 & 0.9239\\
\hline
MIX-EM & 0.3148 & 4.92e-01 & 4.83e-01 & 2.51e-01 & 0.2476 & 3.30e-01 & 3.79e-01 & 0.0745\\
\hline
NM-EM & 0.9704 & 7.53e-01 & 3.74e-01 & 1.98e-01 & 0.9934 & 7.65e-01 & 4.09e-01 & 0.1964\\
\hline
NM-MIX & 0.0897 & 1.50e-03 & 4.83e-06 & 7.96e-10 & 0.0834 & 3.13e-04 & 1.98e-06 & \textbf{0.0000}\\
\hline
\end{tabular}
\end{table}

\hypertarget{phylogenetic-signal-of-diversification-rates-age-and-richness}{%
\subsection{Phylogenetic signal of diversification rates, age, and
richness}\label{phylogenetic-signal-of-diversification-rates-age-and-richness}}

\begin{table}[H]

\caption{\label{tab:unnamed-chunk-11}Phylogenetic signal of the four response variables.}
\centering
\begin{tabular}{l|r}
\hline
Variable & Lambda\\
\hline
r.e0 & 0.475\\
\hline
r.e09 & 0.358\\
\hline
Stem.Age & 1.000\\
\hline
Richness & 0.000\\
\hline
\end{tabular}
\end{table}

\hypertarget{analysis-per-genera-including-all-families}{%
\section{Analysis per genera including all
families}\label{analysis-per-genera-including-all-families}}

\hypertarget{boxplots-1}{%
\subsection{Boxplots}\label{boxplots-1}}

\hypertarget{threshold-50-1}\label{threshold-50-1}}

\begin{figure}
\centering
\includegraphics{supp_mat_files/figure-latex/unnamed-chunk-13-1.pdf}
\caption{Relationship between mycorrhizal type and diversification
rates. a) diversification rate estimated with epsilon = 0 and b)
diversification rate estimated with epsilon = 0.9. AM: Arbuscular
mycorrhiza, EM: Ectomycorrhiza, NM: non-mycorrhizal and MIX (families
with no dominance of any specific mycorrhizal association). The size of
the points indicates the Mycorrhizal Type Diversity Index value for each
lineage, indicating a predominance of larger indices with higher
diversification rates.}
\end{figure}

\hypertarget{threshold-60-1}\label{threshold-60-1}}

\begin{figure}
\centering
\includegraphics{supp_mat_files/figure-latex/unnamed-chunk-14-1.pdf}
\caption{Relationship between mycorrhizal type and diversification
rates. a) diversification rate estimated with epsilon = 0 and b)
diversification rate estimated with epsilon = 0.9. AM: Arbuscular
mycorrhiza, EM: Ectomycorrhiza, NM: non-mycorrhizal and MIX (families
with no dominance of any specific mycorrhizal association). The size of
the points indicates the Mycorrhizal Type Diversity Index value for each
lineage, indicating a predominance of larger indices with higher
diversification rates.}
\end{figure}

\hypertarget{threshold-80-1}\label{threshold-80-1}}

\begin{figure}
\centering
\includegraphics{supp_mat_files/figure-latex/unnamed-chunk-15-1.pdf}
\caption{Relationship between mycorrhizal type and diversification
rates. a) diversification rate estimated with epsilon = 0 and b)
diversification rate estimated with epsilon = 0.9. AM: Arbuscular
mycorrhiza, EM: Ectomycorrhiza, NM: non-mycorrhizal and MIX (families
with no dominance of any specific mycorrhizal association). The size of
the points indicates the Mycorrhizal Type Diversity Index value for each
lineage, indicating a predominance of larger indices with higher
diversification rates.}
\end{figure}

\hypertarget{threshold-100-1}\label{threshold-100-1}}

\begin{figure}
\centering
\includegraphics{supp_mat_files/figure-latex/unnamed-chunk-16-1.pdf}
\caption{Relationship between mycorrhizal type and diversification
rates. a) diversification rate estimated with epsilon = 0 and b)
diversification rate estimated with epsilon = 0.9. AM: Arbuscular
mycorrhiza, EM: Ectomycorrhiza, NM: non-mycorrhizal and MIX (families
with no dominance of any specific mycorrhizal association). The size of
the points indicates the Mycorrhizal Type Diversity Index value for each
lineage, indicating a predominance of larger indices with higher
diversification rates.}
\end{figure}

\pagebreak

\hypertarget{summary-statistics-1}{%
\subsection{Summary statistics}\label{summary-statistics-1}}

\hypertarget{phyanova-2}{%
\subsubsection{phyANOVA}\label{phyanova-2}}

\begin{table}[H]

\caption{\label{tab:unnamed-chunk-17}summary statistics for phyANOVA for both values of epsilon. Significant values are highlighted in bold.}
\centering
\begin{tabular}{l|l|l|l|l}
\hline
Threshold & F.r0 & pvalue.r0 & F.r09 & pvalue.r09\\
\hline
50 & 1.295 & 0.631 & 0.696 & 0.825\\
\hline
60 & 2.842 & 0.244 & 2.153 & 0.402\\
\hline
80 & 8.598 & \textbf{0.005} & 7.033 & \textbf{0.022}\\
\hline
100 & 33.621 & \textbf{0.001} & 37.429 & \textbf{0.001}\\
\hline
\end{tabular}
\end{table}

\hypertarget{standard-anova-2}{%
\subsubsection{Standard ANOVA}\label{standard-anova-2}}

\begin{table}[H]

\caption{\label{tab:unnamed-chunk-18}summary statistics for standard ANOVA for both values of epsilon. Significant values are highlighted in bold.}
\centering
\begin{tabular}{l|l|l|l|l}
\hline
Threshold & F.r0 & pvalue.r0 & F.r09 & pvalue.r09\\
\hline
50 & 1.295 & 0.276 & 0.696 & 0.555\\
\hline
60 & 2.842 & \textbf{0.038} & 2.153 & 0.093\\
\hline
80 & 8.598 & \textbf{0} & 7.033 & \textbf{0}\\
\hline
100 & 33.621 & \textbf{0} & 37.429 & \textbf{0}\\
\hline
\end{tabular}
\end{table}

\hypertarget{posthoc-tests-1}{%
\subsection{Posthoc tests}\label{posthoc-tests-1}}

\hypertarget{phyanova-3}{%
\subsubsection{phyANOVA}\label{phyanova-3}}

\begin{table}[H]

\caption{\label{tab:unnamed-chunk-19}Pairwise Corrected p-values for phyANOVA. Significant values are highlighted in bold.}
\centering
\begin{tabular}{r|l|l|l|l|l|l|l|l|l}
\hline
Threshold & Mycorrhizal.Type & AM.r0 & EM.r0 & MIX.r0 & NM.r0 & AM.r09 & EM.r09 & MIX.r09 & NM.r09\\
\hline
50 & AM & 1 & 1 & 1 & 1 & 1 & 1 & 1 & 1\\
\hline
50 & EM & 1 & 1 & 1 & 1 & 1 & 1 & 1 & 1\\
\hline
50 & MIX & 1 & 1 & 1 & 1 & 1 & 1 & 1 & 1\\
\hline
50 & NM & 1 & 1 & 1 & 1 & 1 & 1 & 1 & 1\\
\hline
60 & AM & 1 & 1 & 0.355 & 1 & 1 & 1 & 0.665 & 1\\
\hline
60 & EM & 1 & 1 & 1 & 1 & 1 & 1 & 1 & 1\\
\hline
60 & MIX & 0.355 & 1 & 1 & 0.33 & 0.665 & 1 & 1 & 0.348\\
\hline
60 & NM & 1 & 1 & 0.33 & 1 & 1 & 1 & 0.348 & 1\\
\hline
80 & AM & 1 & 1 & \textbf{0.01} & 1 & 1 & 1 & \textbf{0.03} & 1\\
\hline
80 & EM & 1 & 1 & 1 & 1 & 1 & 1 & 1 & 1\\
\hline
80 & MIX & \textbf{0.01} & 1 & 1 & \textbf{0.006} & \textbf{0.03} & 1 & 1 & \textbf{0.018}\\
\hline
80 & NM & 1 & 1 & \textbf{0.006} & 1 & 1 & 1 & \textbf{0.018} & 1\\
\hline
100 & AM & 1 & 0.64 & \textbf{0.006} & 0.878 & 1 & 0.796 & \textbf{0.006} & 0.796\\
\hline
100 & EM & 0.64 & 1 & 0.864 & 0.64 & 0.796 & 1 & 0.796 & 0.796\\
\hline
100 & MIX & \textbf{0.006} & 0.864 & 1 & \textbf{0.006} & \textbf{0.006} & 0.796 & 1 & \textbf{0.006}\\
\hline
100 & NM & 0.878 & 0.64 & \textbf{0.006} & 1 & 0.796 & 0.796 & \textbf{0.006} & 1\\
\hline
\end{tabular}
\end{table}

\hypertarget{standard-anova-3}{%
\subsubsection{Standard ANOVA}\label{standard-anova-3}}

\begin{table}[H]

\caption{\label{tab:unnamed-chunk-20}Pairwise Corrected p-values for standard ANOVA. Significant values are highlighted in bold.}
\centering
\begin{tabular}{l|l|l|l|l|l|l|l|l}
\hline
Pairs & 50.r0 & 60.r0 & 80.r0 & 100.r0 & 50.r09 & 60.r09 & 80.r09 & 100.r09\\
\hline
EM.AM & 0.616 & 0.689 & 5.87e-01 & 1.71e-01 & 0.852 & 0.869 & 0.799788 & 2.78e-01\\
\hline
MIX.AM & 0.988 & \textbf{0.032} & 2.12e-05 & \textbf{0.00e+00} & 0.986 & 0.103 & \textbf{0.000379} & \textbf{0.00e+00}\\
\hline
NM.AM & 0.362 & 1.000 & 7.87e-01 & 9.97e-01 & 0.661 & 0.941 & 0.466891 & 9.31e-01\\
\hline
MIX.EM & 0.833 & 0.998 & 9.37e-01 & 6.20e-01 & 0.815 & 0.991 & 0.909145 & 3.67e-01\\
\hline
NM.EM & 0.970 & 0.755 & 3.82e-01 & 2.16e-01 & 0.993 & 0.769 & 0.419083 & 2.26e-01\\
\hline
NM.MIX & 0.907 & 0.186 & 9.22e-04 & 2.98e-07 & 0.763 & 0.139 & \textbf{0.000825} & 8.54e-09\\
\hline
\end{tabular}
\end{table}

\hypertarget{scatterplots}{%
\subsection{Scatterplots}\label{scatterplots}}

\begin{figure}
\centering
\includegraphics{supp_mat_files/figure-latex/unnamed-chunk-21-1.pdf}
\caption{Scatterplots showing the relationship between mycorrhizal
diversity index and diversification rates (a and c), species richness
(b) and age family (d). Diversification rates were estimated with
epsilon = 0 (a) and with epsilon = 0.9 (c). The red and blue lines
indicate the results of a linear model and a phylogenetic generalized
least squares (PGLS) fit, respectively.}
\end{figure}

\begin{table}[H]

\caption{\label{tab:unnamed-chunk-22}Summary statistics for the phylogenetic and parametric regressions using the full genus dataset. Significant values are highlighted in bold.}
\centering
\begin{tabular}{r|l|l|l|l}
\hline
epsilon & pvalue.PGLS & R2.PGLS & pvalue.LM & R2.LM\\
\hline
0.0 & \textbf{4.637e-08} & 0.07343 & \textbf{6.325e-10} & 0.09319\\
\hline
0.9 & \textbf{6.858e-08} & 0.07157 & \textbf{4.512e-09} & 0.08402\\
\hline
\end{tabular}
\end{table}

\hypertarget{family-classification-based-on-different-thresholds}{%
\section{Family classification based on different
thresholds}\label{family-classification-based-on-different-thresholds}}

\begin{table}[H]

\caption{\label{tab:unnamed-chunk-23}Mycorrhizal type assigned to each family based on 4 different percentage thresholds (50, 60, 80 and 100). Significant values are highlighted in bold.}
\centering
\begin{tabular}{l|r|r|r|r|r}
\hline
Threshold & AM & EM & ER & MIX & NM\\
\hline
50 & 296 & 10 & 1 & 23 & 60\\
\hline
60 & 289 & 9 & 1 & 45 & 46\\
\hline
80 & 273 & 9 & 1 & 70 & 37\\
\hline
100 & 231 & 9 & 1 & 116 & 33\\
\hline
\end{tabular}
\end{table}

\pagebreak

\hypertarget{species-level-database}{%
\section{Species-level database}\label{species-level-database}}

\hypertarget{clean-database---excluding-species-with-any-remark}{%
\subsection{Clean database - excluding species with any
remark}\label{clean-database---excluding-species-with-any-remark}}

\hypertarget{boxplots-2}{%
\subsubsection{Boxplots}\label{boxplots-2}}

\hypertarget{threshold-50-2}\label{threshold-50-2}}

\begin{figure}
\centering
\includegraphics{supp_mat_files/figure-latex/unnamed-chunk-25-1.pdf}
\caption{Relationship between mycorrhizal type and diversification
rates. a) diversification rate estimated with epsilon = 0 and b)
diversification rate estimated with epsilon = 0.9. AM: Arbuscular
mycorrhiza, EM: Ectomycorrhiza, NM: non-mycorrhizal and MIX (families
with no dominance of any specific mycorrhizal association). The size of
the points indicates the Mycorrhizal Type Diversity Index value for each
lineage, indicating a predominance of larger indices with higher
diversification rates.}
\end{figure}

\pagebreak

\hypertarget{threshold-60-2}\label{threshold-60-2}}

\begin{figure}
\centering
\includegraphics{supp_mat_files/figure-latex/unnamed-chunk-26-1.pdf}
\caption{Relationship between mycorrhizal type and diversification
rates. a) diversification rate estimated with epsilon = 0 and b)
diversification rate estimated with epsilon = 0.9. AM: Arbuscular
mycorrhiza, EM: Ectomycorrhiza, NM: non-mycorrhizal and MIX (families
with no dominance of any specific mycorrhizal association). The size of
the points indicates the Mycorrhizal Type Diversity Index value for each
lineage, indicating a predominance of larger indices with higher
diversification rates.}
\end{figure}

\pagebreak

\hypertarget{threshold-80-2}\label{threshold-80-2}}

\begin{figure}
\centering
\includegraphics{supp_mat_files/figure-latex/unnamed-chunk-27-1.pdf}
\caption{Relationship between mycorrhizal type and diversification
rates. a) diversification rate estimated with epsilon = 0 and b)
diversification rate estimated with epsilon = 0.9. AM: Arbuscular
mycorrhiza, EM: Ectomycorrhiza, NM: non-mycorrhizal and MIX (families
with no dominance of any specific mycorrhizal association). The size of
the points indicates the Mycorrhizal Type Diversity Index value for each
lineage, indicating a predominance of larger indices with higher
diversification rates.}
\end{figure}

\pagebreak

\hypertarget{threshold-100-2}\label{threshold-100-2}}

\begin{figure}
\centering
\includegraphics{supp_mat_files/figure-latex/unnamed-chunk-28-1.pdf}
\caption{Relationship between mycorrhizal type and diversification
rates. a) diversification rate estimated with epsilon = 0 and b)
diversification rate estimated with epsilon = 0.9. AM: Arbuscular
mycorrhiza, EM: Ectomycorrhiza, NM: non-mycorrhizal and MIX (families
with no dominance of any specific mycorrhizal association). The size of
the points indicates the Mycorrhizal Type Diversity Index value for each
lineage, indicating a predominance of larger indices with higher
diversification rates.}
\end{figure}

\pagebreak

\hypertarget{summary-statistics-2}{%
\subsection{Summary statistics}\label{summary-statistics-2}}

\hypertarget{phyanova-4}{%
\subsubsection{phyANOVA}\label{phyanova-4}}

\begin{table}[H]

\caption{\label{tab:unnamed-chunk-29}summary statistics for phyANOVA for both values of epsilon. Significant values are highlighted in bold.}
\centering
\begin{tabular}{r|l|l|l|l}
\hline
Threshold & F.r0 & pvalue.r0 & F.r09 & pvalue.r09\\
\hline
50 & 0.385 & 0.857 & 0.494 & 0.81\\
\hline
60 & 1.538 & 0.371 & 0.928 & 0.633\\
\hline
80 & 3.784 & 0.084 & 3.262 & 0.117\\
\hline
100 & 24.583 & \textbf{0.001} & 27.097 & \textbf{0.001}\\
\hline
\end{tabular}
\end{table}

\hypertarget{standard-anova-4}{%
\subsubsection{Standard ANOVA}\label{standard-anova-4}}

\begin{table}[H]

\caption{\label{tab:unnamed-chunk-30}summary statistics for phyANOVA for both values of epsilon. Significant values are highlighted in bold.}
\centering
\begin{tabular}{r|l|l|l|l}
\hline
Threshold & F.r0 & pvalue.r0 & F.r09 & pvalue.r09\\
\hline
50 & 0.396 & 0.756 & 0.501 & 0.682\\
\hline
60 & 1.563 & 0.199 & 0.948 & 0.418\\
\hline
80 & 3.529 & \textbf{0.015} & 2.997 & \textbf{0.031}\\
\hline
100 & 23.996 & \textbf{0} & 26.323 & \textbf{0}\\
\hline
\end{tabular}
\end{table}

\hypertarget{posthoc-tests-2}{%
\subsection{Posthoc tests}\label{posthoc-tests-2}}

\hypertarget{phyanova-5}{%
\subsubsection{phyANOVA}\label{phyanova-5}}

\begin{table}[H]

\caption{\label{tab:unnamed-chunk-31}Pairwise Corrected p-values for phyANOVA. Significant values are highlighted in bold.}
\centering
\begin{tabular}{r|l|l|l|l|l|l|l|l|l}
\hline
Threshold & Mycorrhizal.Type & AM.r0 & EM.r0 & MIX.r0 & NM.r0 & AM.r09 & EM.r09 & MIX.r09 & NM.r09\\
\hline
50 & AM & 1 & 1 & 1 & 1 & 1 & 1 & 1 & 1\\
\hline
50 & EM & 1 & 1 & 1 & 1 & 1 & 1 & 1 & 1\\
\hline
50 & MIX & 1 & 1 & 1 & 1 & 1 & 1 & 1 & 1\\
\hline
50 & NM & 1 & 1 & 1 & 1 & 1 & 1 & 1 & 1\\
\hline
60 & AM & 1 & 1 & 0.264 & 1 & 1 & 1 & 0.69 & 1\\
\hline
60 & EM & 1 & 1 & 1 & 1 & 1 & 1 & 1 & 1\\
\hline
60 & MIX & 0.264 & 1 & 1 & 0.995 & 0.69 & 1 & 1 & 1\\
\hline
60 & NM & 1 & 1 & 0.995 & 1 & 1 & 1 & 1 & 1\\
\hline
80 & AM & 1 & 1 & \textbf{0.036} & 1 & 1 & 1 & 0.096 & 1\\
\hline
80 & EM & 1 & 1 & 1 & 1 & 1 & 1 & 1 & 1\\
\hline
80 & MIX & \textbf{0.036} & 1 & 1 & 0.1 & 0.096 & 1 & 1 & 0.096\\
\hline
80 & NM & 1 & 1 & 0.1 & 1 & 1 & 1 & 0.096 & 1\\
\hline
100 & AM & 1 & 0.726 & \textbf{0.006} & 0.726 & 1 & 0.747 & \textbf{0.006} & 0.747\\
\hline
100 & EM & 0.726 & 1 & 0.616 & 0.684 & 0.747 & 1 & 0.376 & 0.747\\
\hline
100 & MIX & \textbf{0.006} & 0.616 & 1 & \textbf{0.006} & \textbf{0.006} & 0.376 & 1 & \textbf{0.006}\\
\hline
100 & NM & 0.726 & 0.684 & \textbf{0.006} & 1 & 0.747 & 0.747 & \textbf{0.006} & 1\\
\hline
\end{tabular}
\end{table}

\hypertarget{standard-anova-5}{%
\subsubsection{Standard ANOVA}\label{standard-anova-5}}

\begin{table}[H]

\caption{\label{tab:unnamed-chunk-32}Pairwise Corrected p-values for standard ANOVA. Significant values are highlighted in bold.}
\centering
\begin{tabular}{l|l|l|l|l|l|l|l|l}
\hline
Types & 50.r0 & 60.r0 & 80.r0 & 100.r0 & 50.r09 & 60.r09 & 80.r09 & 100.r09\\
\hline
EM.AM & 0.964 & 0.992 & 0.9851 & 6.13e-01 & 0.993 & 1.000 & 1.0000 & 7.80e-01\\
\hline
MIX.AM & 0.995 & 0.146 & \textbf{0.0120} & 1.75e-13 & 0.858 & 0.348 & \textbf{0.0358} & 1.22e-13\\
\hline
NM.AM & 0.776 & 0.938 & 0.9112 & 8.08e-01 & 0.831 & 0.974 & 0.7779 & 6.26e-01\\
\hline
MIX.EM & 0.954 & 0.769 & 0.7037 & 2.37e-01 & 0.866 & 0.768 & 0.6219 & 1.06e-01\\
\hline
NM.EM & 1.000 & 1.000 & 0.8968 & 4.11e-01 & 0.996 & 0.993 & 0.9273 & 4.04e-01\\
\hline
NM.MIX & 0.892 & 0.632 & 0.0896 & 3.04e-04 & 0.663 & 0.780 & 0.0810 & 5.05e-05\\
\hline
\end{tabular}
\end{table}

\hypertarget{scatterplots-1}{%
\subsubsection{Scatterplots}\label{scatterplots-1}}

\begin{figure}
\centering
\includegraphics{supp_mat_files/figure-latex/unnamed-chunk-33-1.pdf}
\caption{Scatterplots showing the relationship between mycorrhizal
diversity index and diversification rates (a and c), species richness
(b) and age family (d). Diversification rates were estimated with
epsilon = 0 (a) and with epsilon = 0.9 (c). The red and blue lines
indicate the results of a linear model and a phylogenetic generalized
least squares (PGLS) fit, respectively.}
\end{figure}

\begin{table}[H]

\caption{\label{tab:unnamed-chunk-34}Summary statistics for the phylogenetic and parametric regressions using the full genus dataset. Significant values are highlighted in bold.}
\centering
\begin{tabular}{r|l|l|l|l}
\hline
epsilon & pvalue.PGLS & R2.PGLS & pvalue.LM & R2.LM\\
\hline
0.0 & \textbf{4.063e-07} & 0.08904 & \textbf{1.676e-07} & 0.09458\\
\hline
0.9 & \textbf{3.483e-07} & 0.09007 & \textbf{2.844e-07} & 0.09109\\
\hline
\end{tabular}
\end{table}

\pagebreak

\hypertarget{more-inclusive-database---including-species-with-any-remark}{%
\subsection{More inclusive database - including species with any
remark}\label{more-inclusive-database---including-species-with-any-remark}}

\hypertarget{boxplots-3}{%
\subsubsection{Boxplots}\label{boxplots-3}}

\hypertarget{threshold-50-3}\label{threshold-50-3}}

\begin{figure}
\centering
\includegraphics{supp_mat_files/figure-latex/unnamed-chunk-36-1.pdf}
\caption{Relationship between mycorrhizal type and diversification
rates. a) diversification rate estimated with epsilon = 0 and b)
diversification rate estimated with epsilon = 0.9. AM: Arbuscular
mycorrhiza, EM: Ectomycorrhiza, NM: non-mycorrhizal and MIX (families
with no dominance of any specific mycorrhizal association). The size of
the points indicates the Mycorrhizal Type Diversity Index value for each
lineage, indicating a predominance of larger indices with higher
diversification rates.}
\end{figure}

\pagebreak

\hypertarget{threshold-60-3}\label{threshold-60-3}}

\begin{figure}
\centering
\includegraphics{supp_mat_files/figure-latex/unnamed-chunk-37-1.pdf}
\caption{Relationship between mycorrhizal type and diversification
rates. a) diversification rate estimated with epsilon = 0 and b)
diversification rate estimated with epsilon = 0.9. AM: Arbuscular
mycorrhiza, EM: Ectomycorrhiza, NM: non-mycorrhizal and MIX (families
with no dominance of any specific mycorrhizal association). The size of
the points indicates the Mycorrhizal Type Diversity Index value for each
lineage, indicating a predominance of larger indices with higher
diversification rates.}
\end{figure}

\pagebreak

\hypertarget{threshold-80-3}\label{threshold-80-3}}

\begin{figure}
\centering
\includegraphics{supp_mat_files/figure-latex/unnamed-chunk-38-1.pdf}
\caption{Relationship between mycorrhizal type and diversification
rates. a) diversification rate estimated with epsilon = 0 and b)
diversification rate estimated with epsilon = 0.9. AM: Arbuscular
mycorrhiza, EM: Ectomycorrhiza, NM: non-mycorrhizal and MIX (families
with no dominance of any specific mycorrhizal association). The size of
the points indicates the Mycorrhizal Type Diversity Index value for each
lineage, indicating a predominance of larger indices with higher
diversification rates.}
\end{figure}

\pagebreak

\hypertarget{threshold-100-3}\label{threshold-100-3}}

\begin{figure}
\centering
\includegraphics{supp_mat_files/figure-latex/unnamed-chunk-39-1.pdf}
\caption{Relationship between mycorrhizal type and diversification
rates. a) diversification rate estimated with epsilon = 0 and b)
diversification rate estimated with epsilon = 0.9. AM: Arbuscular
mycorrhiza, EM: Ectomycorrhiza, NM: non-mycorrhizal and MIX (families
with no dominance of any specific mycorrhizal association). The size of
the points indicates the Mycorrhizal Type Diversity Index value for each
lineage, indicating a predominance of larger indices with higher
diversification rates.}
\end{figure}

\pagebreak

\hypertarget{summary-statistics-3}{%
\subsection{Summary statistics}\label{summary-statistics-3}}

\hypertarget{phyanova-6}{%
\subsubsection{phyANOVA}\label{phyanova-6}}

\begin{table}[H]

\caption{\label{tab:unnamed-chunk-40}summary statistics for phyANOVA for both values of epsilon. Significant values are highlighted in bold.}
\centering
\begin{tabular}{r|l|l|l|l}
\hline
Threshold & F.r0 & pvalue.r0 & F.r09 & pvalue.r09\\
\hline
50 & 0.691 & 0.703 & 0.666 & 0.676\\
\hline
60 & 0.511 & 0.787 & 0.324 & 0.869\\
\hline
80 & 8.571 & \textbf{0.001} & 8.425 & \textbf{0.001}\\
\hline
100 & 36.44 & \textbf{0.001} & 43.806 & \textbf{0.001}\\
\hline
\end{tabular}
\end{table}

\hypertarget{standard-anova-6}{%
\subsubsection{Standard ANOVA}\label{standard-anova-6}}

\begin{table}[H]

\caption{\label{tab:unnamed-chunk-41}summary statistics for phyANOVA for both values of epsilon. Significant values are highlighted in bold.}
\centering
\begin{tabular}{r|l|l|l|l}
\hline
Threshold & F.r0 & pvalue.r0 & F.r09 & pvalue.r09\\
\hline
50 & 1.083 & 0.365 & 1.085 & 0.364\\
\hline
60 & 0.953 & 0.434 & 0.831 & 0.506\\
\hline
80 & 7.113 & \textbf{0} & 7.084 & \textbf{0}\\
\hline
100 & 28.22 & \textbf{0} & 34.224 & \textbf{0}\\
\hline
\end{tabular}
\end{table}

\hypertarget{posthoc-tests-3}{%
\subsection{Posthoc tests}\label{posthoc-tests-3}}

\hypertarget{phyanova-7}{%
\subsubsection{phyANOVA}\label{phyanova-7}}

\begin{table}[H]

\caption{\label{tab:unnamed-chunk-42}Pairwise Corrected p-values for phyANOVA. Significant values are highlighted in bold.}
\centering
\begin{tabular}{r|l|l|l|l|l|l|l|l|l}
\hline
Threshold & Mycorrhizal.Type & AM.r0 & EM.r0 & MIX.r0 & NM.r0 & AM.r09 & EM.r09 & MIX.r09 & NM.r09\\
\hline
50 & AM & 1 & 1 & 1 & 1 & 1 & 1 & 1 & 1\\
\hline
50 & EM & 1 & 1 & 1 & 1 & 1 & 1 & 1 & 1\\
\hline
50 & MIX & 1 & 1 & 1 & 0.918 & 1 & 1 & 1 & 0.924\\
\hline
50 & NM & 1 & 1 & 0.918 & 1 & 1 & 1 & 0.924 & 1\\
\hline
60 & AM & 1 & 1 & 1 & 1 & 1 & 1 & 1 & 1\\
\hline
60 & EM & 1 & 1 & 1 & 1 & 1 & 1 & 1 & 1\\
\hline
60 & MIX & 1 & 1 & 1 & 1 & 1 & 1 & 1 & 1\\
\hline
60 & NM & 1 & 1 & 1 & 1 & 1 & 1 & 1 & 1\\
\hline
80 & AM & 1 & 1 & \textbf{0.006} & 1 & 1 & 1 & \textbf{0.006} & 1\\
\hline
80 & EM & 1 & 1 & 1 & 1 & 1 & 1 & 1 & 1\\
\hline
80 & MIX & \textbf{0.006} & 1 & 1 & 0.255 & \textbf{0.006} & 1 & 1 & 0.145\\
\hline
80 & NM & 1 & 1 & 0.255 & 1 & 1 & 1 & 0.145 & 1\\
\hline
100 & AM & 1 & 0.18 & \textbf{0.006} & 0.922 & 1 & 0.2 & \textbf{0.006} & 0.742\\
\hline
100 & EM & 0.18 & 1 & 0.728 & 0.531 & 0.2 & 1 & 0.408 & 0.408\\
\hline
100 & MIX & \textbf{0.006} & 0.728 & 1 & 0.07 & \textbf{0.006} & 0.408 & 1 & \textbf{0.015}\\
\hline
100 & NM & 0.922 & 0.531 & 0.07 & 1 & 0.742 & 0.408 & \textbf{0.015} & 1\\
\hline
\end{tabular}
\end{table}

\hypertarget{standard-anova-7}{%
\subsubsection{Standard ANOVA}\label{standard-anova-7}}

\begin{table}[H]

\caption{\label{tab:unnamed-chunk-43}Pairwise Corrected p-values for standard ANOVA. Significant values are highlighted in bold.}
\centering
\begin{tabular}{l|l|l|l|l|l|l|l|l}
\hline
Types & 50.r0 & 60.r0 & 80.r0 & 100.r0 & 50.r09 & 60.r09 & 80.r09 & 100.r09\\
\hline
EM.AM & 0.990633512621064 & 0.999066929741497 & 0.936868722793265 & 0.0668031153063937 & 0.998335123093735 & 0.999999999960855 & 0.990488217263384 & 0.0842606334652538\\
\hline
ER.AM & 0.59167799644029 & 0.620128699697407 & 0.807809909603096 & 0.999999925753053 & 0.562438984577371 & 0.589820382596277 & 0.783330000390012 & 0.999998779314144\\
\hline
MIX.AM & 0.97044469654804 & 0.804996823389842 & \textbf{9.60945215688902e-06} & \textbf{1.74860126378462e-13} & 0.905942929446393 & 0.873700790534451 & \textbf{1.37389519561104e-05} & \textbf{1.45772283133283e-13}\\
\hline
NM.AM & 0.763273359186079 & 0.958094280569642 & 0.995627685586714 & 0.99997971946529 & 0.849762630865895 & 0.994541887438811 & 0.967977423942333 & 0.997587516127018\\
\hline
ER.EM & 0.5610997307869 & 0.66332560965376 & 0.648683491922557 & 0.526651493003217 & 0.593050347828937 & 0.723495826301868 & 0.735381524309898 & 0.601450526839577\\
\hline
MIX.EM & 0.936742971903572 & 0.995914486646062 & 0.778479187518824 & 0.674060482082851 & 0.928597622187703 & 0.985811513764652 & 0.608411920052696 & 0.363617963156534\\
\hline
NM.EM & 0.997055565542504 & 0.999251070850478 & 0.934538146708019 & 0.582634773582996 & 0.995691997146975 & 0.999061418905608 & 0.928144377636534 & 0.484593518086408\\
\hline
MIX.ER & 0.837524033937929 & 0.406862494515189 & 0.119528054154455 & 0.0500780239809808 & 0.86981652750533 & 0.412694315925131 & 0.112711410601226 & \textbf{0.0289095204882316}\\
\hline
NM.ER & 0.351610960585478 & 0.50056469596149 & 0.979022885971943 & 0.999997287202152 & 0.36786832259415 & 0.56072240034374 & 0.992967230684627 & 0.99861336919152\\
\hline
NM.MIX & 0.68358892614127 & 0.999943164896553 & 0.358349766894571 & 0.102452943553732 & 0.637221057222632 & 0.998717948726249 & 0.236883416673992 & \textbf{0.0303829491519374}\\
\hline
\end{tabular}
\end{table}

\hypertarget{scatterplots-2}{%
\subsubsection{Scatterplots}\label{scatterplots-2}}

\begin{figure}
\centering
\includegraphics{supp_mat_files/figure-latex/unnamed-chunk-44-1.pdf}
\caption{Scatterplots showing the relationship between mycorrhizal
diversity index and diversification rates (a and c), species richness
(b) and age family (d). Diversification rates were estimated with
epsilon = 0 (a) and with epsilon = 0.9 (c). The red and blue lines
indicate the results of a linear model and a phylogenetic generalized
least squares (PGLS) fit, respectively.}
\end{figure}

\begin{table}[H]

\caption{\label{tab:unnamed-chunk-45}Summary statistics for the phylogenetic and parametric regressions using the full genus dataset. Significant values are highlighted in bold.}
\centering
\begin{tabular}{r|l|l|l|l}
\hline
epsilon & pvalue.PGLS & R2.PGLS & pvalue.LM & R2.LM\\
\hline
0.0 & \textbf{5.265e-05} & 0.06444 & \textbf{2.528e-07} & 0.09354\\
\hline
0.9 & \textbf{4.013e-05} & 0.06653 & \textbf{4.977e-07} & 0.08898\\
\hline
\end{tabular}
\end{table}

\hypertarget{adding-randomly-20-of-misassignment-of-mycorrhizal-type}{%
\section{Adding randomly 20\% of misassignment of mycorrhizal
type}\label{adding-randomly-20-of-misassignment-of-mycorrhizal-type}}

\hypertarget{regressions}{%
\subsection{Regressions}\label{regressions}}

\begin{figure}
\centering
\includegraphics{supp_mat_files/figure-latex/unnamed-chunk-47-1.pdf}
\caption{Distribution of (A) p-values and (B) R² values for the same
analyses of the main paper, but using all 50 replicate datasets with
mycorrhizal types for 20\% of species randomly sampled. Dashed vertical
lines represent the corresponding empirical value from the main
analysis.}
\end{figure}

\pagebreak

\hypertarget{anovas}{%
\subsection{ANOVAs}\label{anovas}}

\begin{figure}
\centering
\includegraphics{supp_mat_files/figure-latex/unnamed-chunk-48-1.pdf}
\caption{Distribution of p-values of (A) epsilon = 0 and (B) epsilon =
0.9 for the both phylogenetic and standard ANOVA, but using all 50
replicate datasets with mycorrhizal types for 20\% of species randomly
sampled. Dashed vertical lines represent the corresponding empirical
value from the main analysis.}
\end{figure}


\end{document}
