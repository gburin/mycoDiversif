\documentclass[]{article}
\usepackage{lmodern}
\usepackage{amssymb,amsmath}
\usepackage{ifxetex,ifluatex}
\usepackage{fixltx2e} % provides \textsubscript
\ifnum 0\ifxetex 1\fi\ifluatex 1\fi=0 % if pdftex
  \usepackage[T1]{fontenc}
  \usepackage[utf8]{inputenc}
\else % if luatex or xelatex
  \ifxetex
    \usepackage{mathspec}
  \else
    \usepackage{fontspec}
  \fi
  \defaultfontfeatures{Ligatures=TeX,Scale=MatchLowercase}
\fi
% use upquote if available, for straight quotes in verbatim environments
\IfFileExists{upquote.sty}{\usepackage{upquote}}{}
% use microtype if available
\IfFileExists{microtype.sty}{%
\usepackage{microtype}
\UseMicrotypeSet[protrusion]{basicmath} % disable protrusion for tt fonts
}{}
\usepackage[margin=1in]{geometry}
\usepackage{hyperref}
\hypersetup{unicode=true,
            pdftitle={Supplementary Material - Mujica et al.},
            pdfborder={0 0 0},
            breaklinks=true}
\urlstyle{same}  % don't use monospace font for urls
\usepackage{graphicx,grffile}
\makeatletter
\def\maxwidth{\ifdim\Gin@nat@width>\linewidth\linewidth\else\Gin@nat@width\fi}
\def\maxheight{\ifdim\Gin@nat@height>\textheight\textheight\else\Gin@nat@height\fi}
\makeatother
% Scale images if necessary, so that they will not overflow the page
% margins by default, and it is still possible to overwrite the defaults
% using explicit options in \includegraphics[width, height, ...]{}
\setkeys{Gin}{width=\maxwidth,height=\maxheight,keepaspectratio}
\IfFileExists{parskip.sty}{%
\usepackage{parskip}
}{% else
\setlength{\parindent}{0pt}
\setlength{\parskip}{6pt plus 2pt minus 1pt}
}
\setlength{\emergencystretch}{3em}  % prevent overfull lines
\providecommand{\tightlist}{%
  \setlength{\itemsep}{0pt}\setlength{\parskip}{0pt}}
\setcounter{secnumdepth}{5}
% Redefines (sub)paragraphs to behave more like sections
\ifx\paragraph\undefined\else
\let\oldparagraph\paragraph
\renewcommand{\paragraph}[1]{\oldparagraph{#1}\mbox{}}
\fi
\ifx\subparagraph\undefined\else
\let\oldsubparagraph\subparagraph
\renewcommand{\subparagraph}[1]{\oldsubparagraph{#1}\mbox{}}
\fi

%%% Use protect on footnotes to avoid problems with footnotes in titles
\let\rmarkdownfootnote\footnote%
\def\footnote{\protect\rmarkdownfootnote}

%%% Change title format to be more compact
\usepackage{titling}

% Create subtitle command for use in maketitle
\providecommand{\subtitle}[1]{
  \posttitle{
    \begin{center}\large#1\end{center}
    }
}

\setlength{\droptitle}{-2em}

  \title{Supplementary Material - Mujica et al.}
    \pretitle{\vspace{\droptitle}\centering\huge}
  \posttitle{\par}
    \author{}
    \preauthor{}\postauthor{}
    \date{}
    \predate{}\postdate{}
  
\usepackage{longtable}
\usepackage{float}
\floatplacement{figure}{H}
\floatplacement{table}{H}
\floatplacement{verbatim}{H}

\begin{document}
\maketitle

\hypertarget{phylogenetic-signal-of-mycorrhizal-types}{%
\section{Phylogenetic signal of Mycorrhizal
types}\label{phylogenetic-signal-of-mycorrhizal-types}}

\begin{table}[H]

\caption{\label{tab:unnamed-chunk-2}P-values for test of phylogenetic signal (D) of each mycorrhizal type.}
\centering
\begin{tabular}{r|l|r|r|r|r}
\hline
Threshold & model & AM & EM & MIX & NM\\
\hline
50 & random & 0.000 & 0.000 & 0.842 & 0.000\\
\hline
50 & BM & 0.899 & 0.969 & 0.043 & 0.832\\
\hline
60 & random & 0.000 & 0.000 & 0.031 & 0.000\\
\hline
60 & BM & 0.880 & 0.949 & 0.215 & 0.852\\
\hline
80 & random & 0.000 & 0.000 & 0.000 & 0.000\\
\hline
80 & BM & 0.821 & 0.956 & 0.228 & 0.747\\
\hline
100 & random & 0.000 & 0.000 & 0.247 & 0.000\\
\hline
100 & BM & 0.083 & 0.958 & 0.000 & 0.643\\
\hline
\end{tabular}
\end{table}

\hypertarget{analysis-per-genera-excluding-100-mix-families}{%
\section{Analysis per genera excluding 100\% MIX
families}\label{analysis-per-genera-excluding-100-mix-families}}

\hypertarget{boxplots}{%
\subsection{Boxplots}\label{boxplots}}

\hypertarget{threshold-50}\label{threshold-50}}

\begin{figure}
\centering
\includegraphics{supp_mat_files/figure-latex/unnamed-chunk-3-1.pdf}
\caption{Relationship between mycorrhizal type and diversification
rates. a) diversification rate estimated with ε (relative extinction
fraction) = 0 and b) diversification rate estimated with ε= 0.9. AM:
Arbuscular mycorrhiza, EM: Ectomycorrhiza, NM: non-mycorrhizal and MIX
(families with no dominance of any specific mycorrhizal association).
The size of the points indicates the Mycorrhizal Type Diversity Index
value for each lineage, indicating a predominance of larger indices with
higher diversification rates.}
\end{figure}

\hypertarget{threshold-60}\label{threshold-60}}

\begin{figure}
\centering
\includegraphics{supp_mat_files/figure-latex/unnamed-chunk-4-1.pdf}
\caption{Relationship between mycorrhizal type and diversification
rates. a) diversification rate estimated with ε (relative extinction
fraction) = 0 and b) diversification rate estimated with ε= 0.9. AM:
Arbuscular mycorrhiza, EM: Ectomycorrhiza, NM: non-mycorrhizal and MIX
(families with no dominance of any specific mycorrhizal association).
The size of the points indicates the Mycorrhizal Type Diversity Index
value for each lineage, indicating a predominance of larger indices with
higher diversification rates.}
\end{figure}

\hypertarget{threshold-80}\label{threshold-80}}

\begin{figure}
\centering
\includegraphics{supp_mat_files/figure-latex/unnamed-chunk-5-1.pdf}
\caption{Relationship between mycorrhizal type and diversification
rates. a) diversification rate estimated with ε (relative extinction
fraction) = 0 and b) diversification rate estimated with ε= 0.9. AM:
Arbuscular mycorrhiza, EM: Ectomycorrhiza, NM: non-mycorrhizal and MIX
(families with no dominance of any specific mycorrhizal association).
The size of the points indicates the Mycorrhizal Type Diversity Index
value for each lineage, indicating a predominance of larger indices with
higher diversification rates.}
\end{figure}

\hypertarget{threshold-100}\label{threshold-100}}

\begin{figure}
\centering
\includegraphics{supp_mat_files/figure-latex/unnamed-chunk-6-1.pdf}
\caption{Relationship between mycorrhizal type and diversification
rates. a) diversification rate estimated with ε (relative extinction
fraction) = 0 and b) diversification rate estimated with ε= 0.9. AM:
Arbuscular mycorrhiza, EM: Ectomycorrhiza, NM: non-mycorrhizal and MIX
(families with no dominance of any specific mycorrhizal association).
The size of the points indicates the Mycorrhizal Type Diversity Index
value for each lineage, indicating a predominance of larger indices with
higher diversification rates.}
\end{figure}

\hypertarget{summary-statistics}{%
\subsection{Summary statistics}\label{summary-statistics}}

\hypertarget{phyanova}{%
\subsubsection{phyANOVA}\label{phyanova}}

\begin{table}[H]

\caption{\label{tab:unnamed-chunk-7}Summary statistics for phyANOVA for both values of epsilon}
\centering
\begin{tabular}{r|r|r|r|r}
\hline
Threshold & F\_r0 & pvalue\_r0 & F\_r09 & pvalue\_r09\\
\hline
50 & 3.996351 & 0.089 & 3.146694 & 0.162\\
\hline
60 & 7.255048 & 0.013 & 7.349522 & 0.007\\
\hline
80 & 14.181222 & 0.001 & 13.177448 & 0.001\\
\hline
100 & 44.007242 & 0.001 & 52.475717 & 0.001\\
\hline
\end{tabular}
\end{table}

\hypertarget{standard-anova}{%
\subsubsection{Standard ANOVA}\label{standard-anova}}

\begin{table}[H]

\caption{\label{tab:unnamed-chunk-8}summary statistics for standard ANOVA for both values of epsilon}
\centering
\begin{tabular}{r|r|r|r|r}
\hline
Threshold & F\_r0 & pvalue\_r0 & F\_r09 & pvalue\_r09\\
\hline
50 & 3.996351 & 0.0080666 & 3.146694 & 0.0252019\\
\hline
60 & 7.255048 & 0.0000976 & 7.349522 & 0.0000859\\
\hline
80 & 14.181222 & 0.0000000 & 13.177448 & 0.0000000\\
\hline
100 & 44.007242 & 0.0000000 & 52.475717 & 0.0000000\\
\hline
\end{tabular}
\end{table}

\hypertarget{posthoc-tests}{%
\subsection{Posthoc tests}\label{posthoc-tests}}

\hypertarget{phyanova-1}{%
\subsubsection{phyANOVA}\label{phyanova-1}}

\begin{table}[H]

\caption{\label{tab:unnamed-chunk-9}Pairwise Corrected p-values for phyANOVA}
\centering
\begin{tabular}{r|l|r|r|r|r|r|r|r|r}
\hline
Threshold & Mycorrhizal Type & AM r0 & EM r0 & MIX r0 & NM r0 & AM r09 & EM r09 & MIX r09 & NM r09\\
\hline
50 & AM & 1.000 & 0.858 & 0.042 & 0.747 & 1.000 & 1.000 & 0.048 & 1.000\\
\hline
50 & EM & 0.858 & 1.000 & 0.716 & 0.858 & 1.000 & 1.000 & 0.620 & 1.000\\
\hline
50 & MIX & 0.042 & 0.716 & 1.000 & 0.042 & 0.048 & 0.620 & 1.000 & 0.048\\
\hline
50 & NM & 0.747 & 0.858 & 0.042 & 1.000 & 1.000 & 1.000 & 0.048 & 1.000\\
\hline
60 & AM & 1.000 & 1.000 & 0.006 & 1.000 & 1.000 & 1.000 & 0.006 & 1.000\\
\hline
60 & EM & 1.000 & 1.000 & 1.000 & 1.000 & 1.000 & 1.000 & 0.952 & 1.000\\
\hline
60 & MIX & 0.006 & 1.000 & 1.000 & 0.006 & 0.006 & 0.952 & 1.000 & 0.006\\
\hline
60 & NM & 1.000 & 1.000 & 0.006 & 1.000 & 1.000 & 1.000 & 0.006 & 1.000\\
\hline
80 & AM & 1.000 & 1.000 & 0.006 & 1.000 & 1.000 & 1.000 & 0.006 & 1.000\\
\hline
80 & EM & 1.000 & 1.000 & 1.000 & 1.000 & 1.000 & 1.000 & 1.000 & 1.000\\
\hline
80 & MIX & 0.006 & 1.000 & 1.000 & 0.006 & 0.006 & 1.000 & 1.000 & 0.006\\
\hline
80 & NM & 1.000 & 1.000 & 0.006 & 1.000 & 1.000 & 1.000 & 0.006 & 1.000\\
\hline
100 & AM & 1.000 & 0.552 & 0.006 & 0.868 & 1.000 & 0.567 & 0.006 & 0.627\\
\hline
100 & EM & 0.552 & 1.000 & 0.552 & 0.552 & 0.567 & 1.000 & 0.468 & 0.567\\
\hline
100 & MIX & 0.006 & 0.552 & 1.000 & 0.006 & 0.006 & 0.468 & 1.000 & 0.006\\
\hline
100 & NM & 0.868 & 0.552 & 0.006 & 1.000 & 0.627 & 0.567 & 0.006 & 1.000\\
\hline
\end{tabular}
\end{table}

\hypertarget{standard-anova-1}{%
\subsubsection{Standard ANOVA}\label{standard-anova-1}}

\begin{table}[H]

\caption{\label{tab:unnamed-chunk-10}Pairwise Corrected p-values for standard ANOVA}
\centering
\begin{tabular}{l|r|r|r|r|r|r|r|r}
\hline
Types & 50\% r0 & 60\% r0 & 80\% r0 & 100\% r0 & 50\% r09 & 60\% r09 & 80\% r09 & 100\% r09\\
\hline
EM-AM & 0.6181054 & 0.6863035 & 0.5796271 & 0.1543423 & 0.8535980 & 0.8658222 & 0.7940811 & 0.2458535\\
\hline
MIX-AM & 0.0175076 & 0.0000419 & 0.0000000 & 0.0000000 & 0.0252774 & 0.0000466 & 0.0000001 & 0.0000000\\
\hline
NM-AM & 0.3643504 & 0.9999391 & 0.7825962 & 0.9968473 & 0.6634857 & 0.9395123 & 0.4564819 & 0.9238914\\
\hline
MIX-EM & 0.3148479 & 0.4924541 & 0.4829107 & 0.2510813 & 0.2476270 & 0.3299713 & 0.3786544 & 0.0744958\\
\hline
NM-EM & 0.9703962 & 0.7528214 & 0.3740249 & 0.1976010 & 0.9934429 & 0.7649349 & 0.4085420 & 0.1963633\\
\hline
NM-MIX & 0.0897398 & 0.0015049 & 0.0000048 & 0.0000000 & 0.0833662 & 0.0003135 & 0.0000020 & 0.0000000\\
\hline
\end{tabular}
\end{table}

\hypertarget{phylogenetic-signal-of-diversification-rates-age-and-richness}{%
\subsection{Phylogenetic signal of diversification rates, age, and
richness}\label{phylogenetic-signal-of-diversification-rates-age-and-richness}}

\begin{table}[H]

\caption{\label{tab:unnamed-chunk-11}Phylogenetic signal of the four response variables}
\centering
\begin{tabular}{l|r}
\hline
Variable & Lambda\\
\hline
r (epsilon = 0) & 0.4747942\\
\hline
r (epsilon = 0.9) & 0.3581693\\
\hline
Stem Age & 1.0000000\\
\hline
Richness & 0.0000010\\
\hline
\end{tabular}
\end{table}

\hypertarget{analysis-per-genera-including-all-families}{%
\section{Analysis per genera including all
families}\label{analysis-per-genera-including-all-families}}

\hypertarget{boxplots-1}{%
\subsection{Boxplots}\label{boxplots-1}}

\hypertarget{threshold-50-1}\label{threshold-50-1}}

\begin{figure}
\centering
\includegraphics{supp_mat_files/figure-latex/unnamed-chunk-13-1.pdf}
\caption{Relationship between mycorrhizal type and diversification
rates. a) diversification rate estimated with ε (relative extinction
fraction) = 0 and b) diversification rate estimated with ε= 0.9. AM:
Arbuscular mycorrhiza, EM: Ectomycorrhiza, NM: non-mycorrhizal and MIX
(families with no dominance of any specific mycorrhizal association).
The size of the points indicates the Mycorrhizal Type Diversity Index
value for each lineage, indicating a predominance of larger indices with
higher diversification rates.}
\end{figure}

\hypertarget{threshold-60-1}\label{threshold-60-1}}

\begin{figure}
\centering
\includegraphics{supp_mat_files/figure-latex/unnamed-chunk-14-1.pdf}
\caption{Relationship between mycorrhizal type and diversification
rates. a) diversification rate estimated with ε (relative extinction
fraction) = 0 and b) diversification rate estimated with ε= 0.9. AM:
Arbuscular mycorrhiza, EM: Ectomycorrhiza, NM: non-mycorrhizal and MIX
(families with no dominance of any specific mycorrhizal association).
The size of the points indicates the Mycorrhizal Type Diversity Index
value for each lineage, indicating a predominance of larger indices with
higher diversification rates.}
\end{figure}

\hypertarget{threshold-80-1}\label{threshold-80-1}}

\begin{figure}
\centering
\includegraphics{supp_mat_files/figure-latex/unnamed-chunk-15-1.pdf}
\caption{Relationship between mycorrhizal type and diversification
rates. a) diversification rate estimated with ε (relative extinction
fraction) = 0 and b) diversification rate estimated with ε= 0.9. AM:
Arbuscular mycorrhiza, EM: Ectomycorrhiza, NM: non-mycorrhizal and MIX
(families with no dominance of any specific mycorrhizal association).
The size of the points indicates the Mycorrhizal Type Diversity Index
value for each lineage, indicating a predominance of larger indices with
higher diversification rates.}
\end{figure}

\hypertarget{threshold-100-1}\label{threshold-100-1}}

\begin{figure}
\centering
\includegraphics{supp_mat_files/figure-latex/unnamed-chunk-16-1.pdf}
\caption{Relationship between mycorrhizal type and diversification
rates. a) diversification rate estimated with ε (relative extinction
fraction) = 0 and b) diversification rate estimated with ε= 0.9. AM:
Arbuscular mycorrhiza, EM: Ectomycorrhiza, NM: non-mycorrhizal and MIX
(families with no dominance of any specific mycorrhizal association).
The size of the points indicates the Mycorrhizal Type Diversity Index
value for each lineage, indicating a predominance of larger indices with
higher diversification rates.}
\end{figure}

\hypertarget{summary-statistics-1}{%
\subsection{Summary statistics}\label{summary-statistics-1}}

\hypertarget{phyanova-2}{%
\subsubsection{phyANOVA}\label{phyanova-2}}

\begin{table}[H]

\caption{\label{tab:unnamed-chunk-17}summary statistics for phyANOVA for both values of epsilon}
\centering
\begin{tabular}{r|r|r|r|r}
\hline
Threshold & F\_r0 & pvalue\_r0 & F\_r09 & pvalue\_r09\\
\hline
50 & 1.295208 & 0.631 & 0.6957917 & 0.825\\
\hline
60 & 2.841724 & 0.244 & 2.1533366 & 0.402\\
\hline
80 & 8.598290 & 0.005 & 7.0326320 & 0.022\\
\hline
100 & 33.620981 & 0.001 & 37.4291814 & 0.001\\
\hline
\end{tabular}
\end{table}

\hypertarget{standard-anova-2}{%
\subsubsection{Standard ANOVA}\label{standard-anova-2}}

\begin{table}[H]

\caption{\label{tab:unnamed-chunk-18}summary statistics for phyANOVA for both values of epsilon}
\centering
\begin{tabular}{r|r|r|r|r}
\hline
Threshold & F\_r0 & pvalue\_r0 & F\_r09 & pvalue\_r09\\
\hline
50 & 1.295208 & 0.2757560 & 0.6957917 & 0.5550821\\
\hline
60 & 2.841724 & 0.0377224 & 2.1533366 & 0.0931313\\
\hline
80 & 8.598290 & 0.0000156 & 7.0326320 & 0.0001303\\
\hline
100 & 33.620981 & 0.0000000 & 37.4291814 & 0.0000000\\
\hline
\end{tabular}
\end{table}

\hypertarget{posthoc-tests-1}{%
\subsection{Posthoc tests}\label{posthoc-tests-1}}

\hypertarget{phyanova-3}{%
\subsubsection{phyANOVA}\label{phyanova-3}}

\begin{table}[H]

\caption{\label{tab:unnamed-chunk-19}Pairwise Corrected p-values for phyANOVA}
\centering
\begin{tabular}{r|l|r|r|r|r|r|r|r|r}
\hline
Threshold & Mycorrhizal Type & AM r0 & EM r0 & MIX r0 & NM r0 & AM r09 & EM r09 & MIX r09 & NM r09\\
\hline
50 & AM & 1.000 & 1.000 & 1.000 & 1.000 & 1.000 & 1.000 & 1.000 & 1.000\\
\hline
50 & EM & 1.000 & 1.000 & 1.000 & 1.000 & 1.000 & 1.000 & 1.000 & 1.000\\
\hline
50 & MIX & 1.000 & 1.000 & 1.000 & 1.000 & 1.000 & 1.000 & 1.000 & 1.000\\
\hline
50 & NM & 1.000 & 1.000 & 1.000 & 1.000 & 1.000 & 1.000 & 1.000 & 1.000\\
\hline
60 & AM & 1.000 & 1.000 & 0.355 & 1.000 & 1.000 & 1.000 & 0.665 & 1.000\\
\hline
60 & EM & 1.000 & 1.000 & 1.000 & 1.000 & 1.000 & 1.000 & 1.000 & 1.000\\
\hline
60 & MIX & 0.355 & 1.000 & 1.000 & 0.330 & 0.665 & 1.000 & 1.000 & 0.348\\
\hline
60 & NM & 1.000 & 1.000 & 0.330 & 1.000 & 1.000 & 1.000 & 0.348 & 1.000\\
\hline
80 & AM & 1.000 & 1.000 & 0.010 & 1.000 & 1.000 & 1.000 & 0.030 & 1.000\\
\hline
80 & EM & 1.000 & 1.000 & 1.000 & 1.000 & 1.000 & 1.000 & 1.000 & 1.000\\
\hline
80 & MIX & 0.010 & 1.000 & 1.000 & 0.006 & 0.030 & 1.000 & 1.000 & 0.018\\
\hline
80 & NM & 1.000 & 1.000 & 0.006 & 1.000 & 1.000 & 1.000 & 0.018 & 1.000\\
\hline
100 & AM & 1.000 & 0.640 & 0.006 & 0.878 & 1.000 & 0.796 & 0.006 & 0.796\\
\hline
100 & EM & 0.640 & 1.000 & 0.864 & 0.640 & 0.796 & 1.000 & 0.796 & 0.796\\
\hline
100 & MIX & 0.006 & 0.864 & 1.000 & 0.006 & 0.006 & 0.796 & 1.000 & 0.006\\
\hline
100 & NM & 0.878 & 0.640 & 0.006 & 1.000 & 0.796 & 0.796 & 0.006 & 1.000\\
\hline
\end{tabular}
\end{table}

\hypertarget{standard-anova-3}{%
\subsubsection{Standard ANOVA}\label{standard-anova-3}}

\begin{table}[H]

\caption{\label{tab:unnamed-chunk-20}Pairwise Corrected p-values for standard ANOVA}
\centering
\begin{tabular}{l|r|r|r|r|r|r|r|r}
\hline
Pairs & 50\% r0 & 60\% r0 & 80\% r0 & 100\% r0 & 50\% r09 & 60\% r09 & 80\% r09 & 100\% r09\\
\hline
EM-AM & 0.6162059 & 0.6893711 & 0.5870598 & 0.1711875 & 0.8524170 & 0.8685403 & 0.7997882 & 0.2780424\\
\hline
MIX-AM & 0.9875513 & 0.0319972 & 0.0000212 & 0.0000000 & 0.9859562 & 0.1027975 & 0.0003794 & 0.0000000\\
\hline
NM-AM & 0.3620797 & 0.9999399 & 0.7873166 & 0.9970677 & 0.6612036 & 0.9408275 & 0.4668913 & 0.9314088\\
\hline
MIX-EM & 0.8330187 & 0.9977181 & 0.9370091 & 0.6201825 & 0.8153845 & 0.9912946 & 0.9091448 & 0.3672158\\
\hline
NM-EM & 0.9701924 & 0.7554034 & 0.3823321 & 0.2164235 & 0.9933819 & 0.7692733 & 0.4190835 & 0.2261114\\
\hline
NM-MIX & 0.9074548 & 0.1857903 & 0.0009223 & 0.0000003 & 0.7630198 & 0.1393994 & 0.0008252 & 0.0000000\\
\hline
\end{tabular}
\end{table}

\hypertarget{scatterplots}{%
\subsection{Scatterplots}\label{scatterplots}}

\begin{figure}
\centering
\includegraphics{supp_mat_files/figure-latex/unnamed-chunk-21-1.pdf}
\caption{Scatterplots showing the relationship between mycorrhizal
diversity index and diversification rates (a and c), species richness
(b) and age family (d). Diversification rates were estimated with ε
(relative extinction fraction) = 0 (a) and with ε= 0.9 (c). The red and
blue lines indicate the results of a linear model and a phylogenetic
generalized least squares (PGLS) fit, respectively.}
\end{figure}

\begin{longtable}{r|r|r|r|r}
\caption{\label{tab:unnamed-chunk-22}Summary statistics for the phylogenetic and parametric regressions using the full genus dataset.}\\
\hline
epsilon & pvalue\_PGLS & R2\_PGLS & pvalue\_LM & R2\_LM\\
\hline
0.0 & 0e+00 & 0.07343 & 0 & 0.09319\\
\hline
0.9 & 1e-07 & 0.07157 & 0 & 0.08402\\
\hline
\end{longtable}

\hypertarget{family-classification-based-on-different-thresholds}{%
\section{Family classification based on different
thresholds}\label{family-classification-based-on-different-thresholds}}

\begin{longtable}{l|r|r|r|r|r}
\caption{\label{tab:unnamed-chunk-23}Mycorrhizal type assigned to each family based on 4 different percentage thresholds (50, 60, 80 and 100)}\\
\hline
Threshold & AM & EM & ER & MIX & NM\\
\hline
50\% & 296 & 10 & 1 & 23 & 60\\
\hline
60\% & 289 & 9 & 1 & 45 & 46\\
\hline
80\% & 273 & 9 & 1 & 70 & 37\\
\hline
100\% & 231 & 9 & 1 & 116 & 33\\
\hline
\end{longtable}

\hypertarget{species-level-database}{%
\section{Species-level database}\label{species-level-database}}

\hypertarget{clean-database---excluding-species-with-any-remark}{%
\subsection{Clean database - excluding species with any
remark}\label{clean-database---excluding-species-with-any-remark}}

\hypertarget{boxplots-2}{%
\subsubsection{Boxplots}\label{boxplots-2}}

\hypertarget{threshold-50-2}\label{threshold-50-2}}

\begin{figure}
\centering
\includegraphics{supp_mat_files/figure-latex/unnamed-chunk-25-1.pdf}
\caption{Relationship between mycorrhizal type and diversification
rates. a) diversification rate estimated with ε (relative extinction
fraction) = 0 and b) diversification rate estimated with ε= 0.9. AM:
Arbuscular mycorrhiza, EM: Ectomycorrhiza, NM: non-mycorrhizal and MIX
(families with no dominance of any specific mycorrhizal association).
The size of the points indicates the Mycorrhizal Type Diversity Index
value for each lineage, indicating a predominance of larger indices with
higher diversification rates.}
\end{figure}

\hypertarget{threshold-60-2}\label{threshold-60-2}}

\begin{figure}
\centering
\includegraphics{supp_mat_files/figure-latex/unnamed-chunk-26-1.pdf}
\caption{Relationship between mycorrhizal type and diversification
rates. a) diversification rate estimated with ε (relative extinction
fraction) = 0 and b) diversification rate estimated with ε= 0.9. AM:
Arbuscular mycorrhiza, EM: Ectomycorrhiza, NM: non-mycorrhizal and MIX
(families with no dominance of any specific mycorrhizal association).
The size of the points indicates the Mycorrhizal Type Diversity Index
value for each lineage, indicating a predominance of larger indices with
higher diversification rates.}
\end{figure}

\hypertarget{threshold-80-2}\label{threshold-80-2}}

\begin{figure}
\centering
\includegraphics{supp_mat_files/figure-latex/unnamed-chunk-27-1.pdf}
\caption{Relationship between mycorrhizal type and diversification
rates. a) diversification rate estimated with ε (relative extinction
fraction) = 0 and b) diversification rate estimated with ε= 0.9. AM:
Arbuscular mycorrhiza, EM: Ectomycorrhiza, NM: non-mycorrhizal and MIX
(families with no dominance of any specific mycorrhizal association).
The size of the points indicates the Mycorrhizal Type Diversity Index
value for each lineage, indicating a predominance of larger indices with
higher diversification rates.}
\end{figure}

\hypertarget{threshold-100-2}\label{threshold-100-2}}

\begin{figure}
\centering
\includegraphics{supp_mat_files/figure-latex/unnamed-chunk-28-1.pdf}
\caption{Relationship between mycorrhizal type and diversification
rates. a) diversification rate estimated with ε (relative extinction
fraction) = 0 and b) diversification rate estimated with ε= 0.9. AM:
Arbuscular mycorrhiza, EM: Ectomycorrhiza, NM: non-mycorrhizal and MIX
(families with no dominance of any specific mycorrhizal association).
The size of the points indicates the Mycorrhizal Type Diversity Index
value for each lineage, indicating a predominance of larger indices with
higher diversification rates.}
\end{figure}

\hypertarget{summary-statistics-2}{%
\subsection{Summary statistics}\label{summary-statistics-2}}

\hypertarget{phyanova-4}{%
\subsubsection{phyANOVA}\label{phyanova-4}}

\begin{table}[H]

\caption{\label{tab:unnamed-chunk-29}summary statistics for phyANOVA for both values of epsilon}
\centering
\begin{tabular}{r|r|r|r|r}
\hline
Threshold & F\_r0 & pvalue\_r0 & F\_r09 & pvalue\_r09\\
\hline
50 & 0.3851399 & 0.857 & 0.4939560 & 0.810\\
\hline
60 & 1.5383760 & 0.371 & 0.9275641 & 0.633\\
\hline
80 & 3.7844578 & 0.084 & 3.2616354 & 0.117\\
\hline
100 & 24.5826918 & 0.001 & 27.0967625 & 0.001\\
\hline
\end{tabular}
\end{table}

\hypertarget{standard-anova-4}{%
\subsubsection{Standard ANOVA}\label{standard-anova-4}}

\begin{table}[H]

\caption{\label{tab:unnamed-chunk-30}summary statistics for phyANOVA for both values of epsilon}
\centering
\begin{tabular}{r|r|r|r|r}
\hline
Threshold & F\_r0 & pvalue\_r0 & F\_r09 & pvalue\_r09\\
\hline
50 & 0.3955297 & 0.7563236 & 0.5009815 & 0.6819207\\
\hline
60 & 1.5633983 & 0.1986397 & 0.9479786 & 0.4179219\\
\hline
80 & 3.5294456 & 0.0154472 & 2.9970343 & 0.0312388\\
\hline
100 & 23.9958898 & 0.0000000 & 26.3234262 & 0.0000000\\
\hline
\end{tabular}
\end{table}

\hypertarget{posthoc-tests-2}{%
\subsection{Posthoc tests}\label{posthoc-tests-2}}

\hypertarget{phyanova-5}{%
\subsubsection{phyANOVA}\label{phyanova-5}}

\begin{table}[H]

\caption{\label{tab:unnamed-chunk-31}Pairwise Corrected p-values for phyANOVA}
\centering
\begin{tabular}{r|l|r|r|r|r|r|r|r|r}
\hline
Threshold & Mycorrhizal Type & AM r0 & EM r0 & MIX r0 & NM r0 & AM r09 & EM r09 & MIX r09 & NM r09\\
\hline
50 & AM & 1.000 & 1.000 & 1.000 & 1.000 & 1.000 & 1.000 & 1.000 & 1.000\\
\hline
50 & EM & 1.000 & 1.000 & 1.000 & 1.000 & 1.000 & 1.000 & 1.000 & 1.000\\
\hline
50 & MIX & 1.000 & 1.000 & 1.000 & 1.000 & 1.000 & 1.000 & 1.000 & 1.000\\
\hline
50 & NM & 1.000 & 1.000 & 1.000 & 1.000 & 1.000 & 1.000 & 1.000 & 1.000\\
\hline
60 & AM & 1.000 & 1.000 & 0.264 & 1.000 & 1.000 & 1.000 & 0.690 & 1.000\\
\hline
60 & EM & 1.000 & 1.000 & 1.000 & 1.000 & 1.000 & 1.000 & 1.000 & 1.000\\
\hline
60 & MIX & 0.264 & 1.000 & 1.000 & 0.995 & 0.690 & 1.000 & 1.000 & 1.000\\
\hline
60 & NM & 1.000 & 1.000 & 0.995 & 1.000 & 1.000 & 1.000 & 1.000 & 1.000\\
\hline
80 & AM & 1.000 & 1.000 & 0.036 & 1.000 & 1.000 & 1.000 & 0.096 & 1.000\\
\hline
80 & EM & 1.000 & 1.000 & 1.000 & 1.000 & 1.000 & 1.000 & 1.000 & 1.000\\
\hline
80 & MIX & 0.036 & 1.000 & 1.000 & 0.100 & 0.096 & 1.000 & 1.000 & 0.096\\
\hline
80 & NM & 1.000 & 1.000 & 0.100 & 1.000 & 1.000 & 1.000 & 0.096 & 1.000\\
\hline
100 & AM & 1.000 & 0.726 & 0.006 & 0.726 & 1.000 & 0.747 & 0.006 & 0.747\\
\hline
100 & EM & 0.726 & 1.000 & 0.616 & 0.684 & 0.747 & 1.000 & 0.376 & 0.747\\
\hline
100 & MIX & 0.006 & 0.616 & 1.000 & 0.006 & 0.006 & 0.376 & 1.000 & 0.006\\
\hline
100 & NM & 0.726 & 0.684 & 0.006 & 1.000 & 0.747 & 0.747 & 0.006 & 1.000\\
\hline
\end{tabular}
\end{table}

\hypertarget{standard-anova-5}{%
\subsubsection{Standard ANOVA}\label{standard-anova-5}}

\begin{table}[H]

\caption{\label{tab:unnamed-chunk-32}Pairwise Corrected p-values for standard ANOVA}
\centering
\begin{tabular}{l|r|r|r|r|r|r|r|r}
\hline
Types & 50\% r0 & 60\% r0 & 80\% r0 & 100\% r0 & 50\% r09 & 60\% r09 & 80\% r09 & 100\% r09\\
\hline
EM-AM & 0.9640138 & 0.9924574 & 0.9850669 & 0.6125728 & 0.9933364 & 0.9999863 & 0.9999926 & 0.7802549\\
\hline
MIX-AM & 0.9945230 & 0.1464660 & 0.0120044 & 0.0000000 & 0.8583738 & 0.3481935 & 0.0358281 & 0.0000000\\
\hline
NM-AM & 0.7758750 & 0.9382159 & 0.9111597 & 0.8079435 & 0.8308462 & 0.9735179 & 0.7778657 & 0.6260774\\
\hline
MIX-EM & 0.9540250 & 0.7689916 & 0.7036769 & 0.2367099 & 0.8663578 & 0.7681943 & 0.6218547 & 0.1059495\\
\hline
NM-EM & 0.9997752 & 0.9998371 & 0.8968196 & 0.4108561 & 0.9964156 & 0.9934415 & 0.9273026 & 0.4041777\\
\hline
NM-MIX & 0.8923413 & 0.6318123 & 0.0896241 & 0.0003035 & 0.6631972 & 0.7804947 & 0.0810474 & 0.0000505\\
\hline
\end{tabular}
\end{table}

\hypertarget{scatterplots-1}{%
\subsubsection{Scatterplots}\label{scatterplots-1}}

\begin{figure}
\centering
\includegraphics{supp_mat_files/figure-latex/unnamed-chunk-33-1.pdf}
\caption{Scatterplots showing the relationship between mycorrhizal
diversity index and diversification rates (a and c), species richness
(b) and age family (d). Diversification rates were estimated with ε
(relative extinction fraction) = 0 (a) and with ε= 0.9 (c). The red and
blue lines indicate the results of a linear model and a phylogenetic
generalized least squares (PGLS) fit, respectively.}
\end{figure}

\begin{longtable}{r|r|r|r|r}
\caption{\label{tab:unnamed-chunk-34}Summary statistics for the phylogenetic and parametric regressions using the full genus dataset.}\\
\hline
epsilon & pvalue\_PGLS & R2\_PGLS & pvalue\_LM & R2\_LM\\
\hline
0.0 & 4e-07 & 0.08904 & 2e-07 & 0.09458\\
\hline
0.9 & 3e-07 & 0.09007 & 3e-07 & 0.09109\\
\hline
\end{longtable}

\hypertarget{more-inclusive-database---including-species-with-any-remark}{%
\subsection{More inclusive database - including species with any
remark}\label{more-inclusive-database---including-species-with-any-remark}}

\hypertarget{boxplots-3}{%
\subsubsection{Boxplots}\label{boxplots-3}}

\hypertarget{threshold-50-3}\label{threshold-50-3}}

\begin{figure}
\centering
\includegraphics{supp_mat_files/figure-latex/unnamed-chunk-36-1.pdf}
\caption{Relationship between mycorrhizal type and diversification
rates. a) diversification rate estimated with ε (relative extinction
fraction) = 0 and b) diversification rate estimated with ε= 0.9. AM:
Arbuscular mycorrhiza, EM: Ectomycorrhiza, NM: non-mycorrhizal and MIX
(families with no dominance of any specific mycorrhizal association).
The size of the points indicates the Mycorrhizal Type Diversity Index
value for each lineage, indicating a predominance of larger indices with
higher diversification rates.}
\end{figure}

\hypertarget{threshold-60-3}\label{threshold-60-3}}

\begin{figure}
\centering
\includegraphics{supp_mat_files/figure-latex/unnamed-chunk-37-1.pdf}
\caption{Relationship between mycorrhizal type and diversification
rates. a) diversification rate estimated with ε (relative extinction
fraction) = 0 and b) diversification rate estimated with ε= 0.9. AM:
Arbuscular mycorrhiza, EM: Ectomycorrhiza, NM: non-mycorrhizal and MIX
(families with no dominance of any specific mycorrhizal association).
The size of the points indicates the Mycorrhizal Type Diversity Index
value for each lineage, indicating a predominance of larger indices with
higher diversification rates.}
\end{figure}

\hypertarget{threshold-80-3}\label{threshold-80-3}}

\begin{figure}
\centering
\includegraphics{supp_mat_files/figure-latex/unnamed-chunk-38-1.pdf}
\caption{Relationship between mycorrhizal type and diversification
rates. a) diversification rate estimated with ε (relative extinction
fraction) = 0 and b) diversification rate estimated with ε= 0.9. AM:
Arbuscular mycorrhiza, EM: Ectomycorrhiza, NM: non-mycorrhizal and MIX
(families with no dominance of any specific mycorrhizal association).
The size of the points indicates the Mycorrhizal Type Diversity Index
value for each lineage, indicating a predominance of larger indices with
higher diversification rates.}
\end{figure}

\hypertarget{threshold-100-3}\label{threshold-100-3}}

\begin{figure}
\centering
\includegraphics{supp_mat_files/figure-latex/unnamed-chunk-39-1.pdf}
\caption{Relationship between mycorrhizal type and diversification
rates. a) diversification rate estimated with ε (relative extinction
fraction) = 0 and b) diversification rate estimated with ε= 0.9. AM:
Arbuscular mycorrhiza, EM: Ectomycorrhiza, NM: non-mycorrhizal and MIX
(families with no dominance of any specific mycorrhizal association).
The size of the points indicates the Mycorrhizal Type Diversity Index
value for each lineage, indicating a predominance of larger indices with
higher diversification rates.}
\end{figure}

\hypertarget{summary-statistics-3}{%
\subsection{Summary statistics}\label{summary-statistics-3}}

\hypertarget{phyanova-6}{%
\subsubsection{phyANOVA}\label{phyanova-6}}

\begin{table}[H]

\caption{\label{tab:unnamed-chunk-40}summary statistics for phyANOVA for both values of epsilon}
\centering
\begin{tabular}{r|r|r|r|r}
\hline
Threshold & F\_r0 & pvalue\_r0 & F\_r09 & pvalue\_r09\\
\hline
50 & 0.6913713 & 0.703 & 0.6662580 & 0.676\\
\hline
60 & 0.5110525 & 0.787 & 0.3237887 & 0.869\\
\hline
80 & 8.5708941 & 0.001 & 8.4253115 & 0.001\\
\hline
100 & 36.4395067 & 0.001 & 43.8064306 & 0.001\\
\hline
\end{tabular}
\end{table}

\hypertarget{standard-anova-6}{%
\subsubsection{Standard ANOVA}\label{standard-anova-6}}

\begin{table}[H]

\caption{\label{tab:unnamed-chunk-41}summary statistics for phyANOVA for both values of epsilon}
\centering
\begin{tabular}{r|r|r|r|r}
\hline
Threshold & F\_r0 & pvalue\_r0 & F\_r09 & pvalue\_r09\\
\hline
50 & 1.0830611 & 0.3653162 & 1.0848972 & 0.3644122\\
\hline
60 & 0.9527973 & 0.4340430 & 0.8309617 & 0.5064990\\
\hline
80 & 7.1127102 & 0.0000190 & 7.0839219 & 0.0000200\\
\hline
100 & 28.2200847 & 0.0000000 & 34.2240624 & 0.0000000\\
\hline
\end{tabular}
\end{table}

\hypertarget{posthoc-tests-3}{%
\subsection{Posthoc tests}\label{posthoc-tests-3}}

\hypertarget{phyanova-7}{%
\subsubsection{phyANOVA}\label{phyanova-7}}

\begin{table}[H]

\caption{\label{tab:unnamed-chunk-42}Pairwise Corrected p-values for phyANOVA}
\centering
\begin{tabular}{r|l|r|r|r|r|r|r|r|r}
\hline
Threshold & Mycorrhizal Type & AM r0 & EM r0 & MIX r0 & NM r0 & AM r09 & EM r09 & MIX r09 & NM r09\\
\hline
50 & AM & 1.000 & 1.000 & 1.000 & 1.000 & 1.000 & 1.000 & 1.000 & 1.000\\
\hline
50 & EM & 1.000 & 1.000 & 1.000 & 1.000 & 1.000 & 1.000 & 1.000 & 1.000\\
\hline
50 & MIX & 1.000 & 1.000 & 1.000 & 0.918 & 1.000 & 1.000 & 1.000 & 0.924\\
\hline
50 & NM & 1.000 & 1.000 & 0.918 & 1.000 & 1.000 & 1.000 & 0.924 & 1.000\\
\hline
60 & AM & 1.000 & 1.000 & 1.000 & 1.000 & 1.000 & 1.000 & 1.000 & 1.000\\
\hline
60 & EM & 1.000 & 1.000 & 1.000 & 1.000 & 1.000 & 1.000 & 1.000 & 1.000\\
\hline
60 & MIX & 1.000 & 1.000 & 1.000 & 1.000 & 1.000 & 1.000 & 1.000 & 1.000\\
\hline
60 & NM & 1.000 & 1.000 & 1.000 & 1.000 & 1.000 & 1.000 & 1.000 & 1.000\\
\hline
80 & AM & 1.000 & 1.000 & 0.006 & 1.000 & 1.000 & 1.000 & 0.006 & 1.000\\
\hline
80 & EM & 1.000 & 1.000 & 1.000 & 1.000 & 1.000 & 1.000 & 1.000 & 1.000\\
\hline
80 & MIX & 0.006 & 1.000 & 1.000 & 0.255 & 0.006 & 1.000 & 1.000 & 0.145\\
\hline
80 & NM & 1.000 & 1.000 & 0.255 & 1.000 & 1.000 & 1.000 & 0.145 & 1.000\\
\hline
100 & AM & 1.000 & 0.180 & 0.006 & 0.922 & 1.000 & 0.200 & 0.006 & 0.742\\
\hline
100 & EM & 0.180 & 1.000 & 0.728 & 0.531 & 0.200 & 1.000 & 0.408 & 0.408\\
\hline
100 & MIX & 0.006 & 0.728 & 1.000 & 0.070 & 0.006 & 0.408 & 1.000 & 0.015\\
\hline
100 & NM & 0.922 & 0.531 & 0.070 & 1.000 & 0.742 & 0.408 & 0.015 & 1.000\\
\hline
\end{tabular}
\end{table}

\hypertarget{standard-anova-7}{%
\subsubsection{Standard ANOVA}\label{standard-anova-7}}

\begin{table}[H]

\caption{\label{tab:unnamed-chunk-43}Pairwise Corrected p-values for standard ANOVA}
\centering
\begin{tabular}{l|r|r|r|r|r|r|r|r}
\hline
Types & 50\% r0 & 60\% r0 & 80\% r0 & 100\% r0 & 50\% r09 & 60\% r09 & 80\% r09 & 100\% r09\\
\hline
EM-AM & 0.9906335 & 0.9990669 & 0.9368687 & 0.0668031 & 0.9983351 & 1.0000000 & 0.9904882 & 0.0842606\\
\hline
ER-AM & 0.5916780 & 0.6201287 & 0.8078099 & 0.9999999 & 0.5624390 & 0.5898204 & 0.7833300 & 0.9999988\\
\hline
MIX-AM & 0.9704447 & 0.8049968 & 0.0000096 & 0.0000000 & 0.9059429 & 0.8737008 & 0.0000137 & 0.0000000\\
\hline
NM-AM & 0.7632734 & 0.9580943 & 0.9956277 & 0.9999797 & 0.8497626 & 0.9945419 & 0.9679774 & 0.9975875\\
\hline
ER-EM & 0.5610997 & 0.6633256 & 0.6486835 & 0.5266515 & 0.5930503 & 0.7234958 & 0.7353815 & 0.6014505\\
\hline
MIX-EM & 0.9367430 & 0.9959145 & 0.7784792 & 0.6740605 & 0.9285976 & 0.9858115 & 0.6084119 & 0.3636180\\
\hline
NM-EM & 0.9970556 & 0.9992511 & 0.9345381 & 0.5826348 & 0.9956920 & 0.9990614 & 0.9281444 & 0.4845935\\
\hline
MIX-ER & 0.8375240 & 0.4068625 & 0.1195281 & 0.0500780 & 0.8698165 & 0.4126943 & 0.1127114 & 0.0289095\\
\hline
NM-ER & 0.3516110 & 0.5005647 & 0.9790229 & 0.9999973 & 0.3678683 & 0.5607224 & 0.9929672 & 0.9986134\\
\hline
NM-MIX & 0.6835889 & 0.9999432 & 0.3583498 & 0.1024529 & 0.6372211 & 0.9987179 & 0.2368834 & 0.0303829\\
\hline
\end{tabular}
\end{table}

\hypertarget{scatterplots-2}{%
\subsubsection{Scatterplots}\label{scatterplots-2}}

\begin{figure}
\centering
\includegraphics{supp_mat_files/figure-latex/unnamed-chunk-44-1.pdf}
\caption{Scatterplots showing the relationship between mycorrhizal
diversity index and diversification rates (a and c), species richness
(b) and age family (d). Diversification rates were estimated with ε
(relative extinction fraction) = 0 (a) and with ε= 0.9 (c). The red and
blue lines indicate the results of a linear model and a phylogenetic
generalized least squares (PGLS) fit, respectively.}
\end{figure}

\begin{longtable}{r|r|r|r|r}
\caption{\label{tab:unnamed-chunk-45}Summary statistics for the phylogenetic and parametric regressions using the full genus dataset.}\\
\hline
epsilon & pvalue\_PGLS & R2\_PGLS & pvalue\_LM & R2\_LM\\
\hline
0.0 & 5.26e-05 & 0.06444 & 3e-07 & 0.09354\\
\hline
0.9 & 4.01e-05 & 0.06653 & 5e-07 & 0.08898\\
\hline
\end{longtable}

\hypertarget{adding-randomly-20-of-misassignment-of-mycorrhizal-type}{%
\section{Adding randomly 20\% of misassignment of mycorrhizal
type}\label{adding-randomly-20-of-misassignment-of-mycorrhizal-type}}

\hypertarget{regressions}{%
\subsection{Regressions}\label{regressions}}

\begin{figure}
\centering
\includegraphics{supp_mat_files/figure-latex/unnamed-chunk-47-1.pdf}
\caption{Distribution of (A) p-values and (B) R² values for the same
analyses of the main paper, but using all 50 replicate datasets with
mycorrhizal types for 20\% of species randomly sampled. Dashed vertical
lines represent the corresponding empirical value from the main
analysis.}
\end{figure}

\pagebreak

\hypertarget{anovas}{%
\subsection{ANOVAs}\label{anovas}}

\begin{figure}
\centering
\includegraphics{supp_mat_files/figure-latex/unnamed-chunk-48-1.pdf}
\caption{Distribution of p-values of (A) epsilon = 0 and (B) epsilon =
0.9 for the both phylogenetic and standard ANOVA, but using all 50
replicate datasets with mycorrhizal types for 20\% of species randomly
sampled. Dashed vertical lines represent the corresponding empirical
value from the main analysis.}
\end{figure}


\end{document}
