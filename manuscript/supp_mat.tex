\documentclass[]{article}
\usepackage{lmodern}
\usepackage{amssymb,amsmath}
\usepackage{ifxetex,ifluatex}
\usepackage{fixltx2e} % provides \textsubscript
\ifnum 0\ifxetex 1\fi\ifluatex 1\fi=0 % if pdftex
  \usepackage[T1]{fontenc}
  \usepackage[utf8]{inputenc}
\else % if luatex or xelatex
  \ifxetex
    \usepackage{mathspec}
  \else
    \usepackage{fontspec}
  \fi
  \defaultfontfeatures{Ligatures=TeX,Scale=MatchLowercase}
\fi
% use upquote if available, for straight quotes in verbatim environments
\IfFileExists{upquote.sty}{\usepackage{upquote}}{}
% use microtype if available
\IfFileExists{microtype.sty}{%
\usepackage{microtype}
\UseMicrotypeSet[protrusion]{basicmath} % disable protrusion for tt fonts
}{}
\usepackage[margin=1in]{geometry}
\usepackage{hyperref}
\hypersetup{unicode=true,
            pdftitle={Supplementary Material - Mujica et al.},
            pdfborder={0 0 0},
            breaklinks=true}
\urlstyle{same}  % don't use monospace font for urls
\usepackage{graphicx,grffile}
\makeatletter
\def\maxwidth{\ifdim\Gin@nat@width>\linewidth\linewidth\else\Gin@nat@width\fi}
\def\maxheight{\ifdim\Gin@nat@height>\textheight\textheight\else\Gin@nat@height\fi}
\makeatother
% Scale images if necessary, so that they will not overflow the page
% margins by default, and it is still possible to overwrite the defaults
% using explicit options in \includegraphics[width, height, ...]{}
\setkeys{Gin}{width=\maxwidth,height=\maxheight,keepaspectratio}
\IfFileExists{parskip.sty}{%
\usepackage{parskip}
}{% else
\setlength{\parindent}{0pt}
\setlength{\parskip}{6pt plus 2pt minus 1pt}
}
\setlength{\emergencystretch}{3em}  % prevent overfull lines
\providecommand{\tightlist}{%
  \setlength{\itemsep}{0pt}\setlength{\parskip}{0pt}}
\setcounter{secnumdepth}{5}
% Redefines (sub)paragraphs to behave more like sections
\ifx\paragraph\undefined\else
\let\oldparagraph\paragraph
\renewcommand{\paragraph}[1]{\oldparagraph{#1}\mbox{}}
\fi
\ifx\subparagraph\undefined\else
\let\oldsubparagraph\subparagraph
\renewcommand{\subparagraph}[1]{\oldsubparagraph{#1}\mbox{}}
\fi

%%% Use protect on footnotes to avoid problems with footnotes in titles
\let\rmarkdownfootnote\footnote%
\def\footnote{\protect\rmarkdownfootnote}

%%% Change title format to be more compact
\usepackage{titling}

% Create subtitle command for use in maketitle
\providecommand{\subtitle}[1]{
  \posttitle{
    \begin{center}\large#1\end{center}
    }
}

\setlength{\droptitle}{-2em}

  \title{Supplementary Material - Mujica et al.}
    \pretitle{\vspace{\droptitle}\centering\huge}
  \posttitle{\par}
    \author{}
    \preauthor{}\postauthor{}
    \date{}
    \predate{}\postdate{}
  
\usepackage{longtable}
\usepackage{float}
\floatplacement{figure}{H}
\floatplacement{table}{H}
\floatplacement{verbatim}{H}

\begin{document}
\maketitle

\hypertarget{adding-randomly-20-of-misassignment-of-mycorrhizal-type}{%
\section{Adding randomly 20\% of misassignment of mycorrhizal
type}\label{adding-randomly-20-of-misassignment-of-mycorrhizal-type}}

\hypertarget{analysis-per-genera-including-all-families}{%
\section{Analysis per genera including all
families}\label{analysis-per-genera-including-all-families}}

\hypertarget{boxplots}{%
\subsection{Boxplots}\label{boxplots}}

\begin{figure}
\centering
\includegraphics{supp_mat_files/figure-latex/unnamed-chunk-2-1.pdf}
\caption{Relationship between mycorrhizal type and diversification
rates. a) diversification rate estimated with ε (relative extinction
fraction) = 0 and b) diversification rate estimated with ε= 0.9. AM:
Arbuscular mycorrhiza, EM: Ectomycorrhiza, NM: non-mycorrhizal and MIX
(families with no dominance of any specific mycorrhizal association).
The size of the points indicates the Mycorrhizal Type Diversity Index
value for each lineage, indicating a predominance of larger indices with
higher diversification rates.}
\end{figure}

\hypertarget{scatterplots}{%
\subsection{Scatterplots}\label{scatterplots}}

\begin{figure}
\centering
\includegraphics{supp_mat_files/figure-latex/unnamed-chunk-3-1.pdf}
\caption{Scatterplots showing the relationship between mycorrhizal
diversity index and diversification rates (a and c), species richness
(b) and age family (d). Diversification rates were estimated with ε
(relative extinction fraction) = 0 (a) and with ε= 0.9 (c). The red and
blue lines indicate the results of a linear model and a phylogenetic
generalized least squares (PGLS) fit, respectively.}
\end{figure}

\hypertarget{different-thresholds-for-main-mycorrhizal-type}{%
\section{Different thresholds for main mycorrhizal
type}\label{different-thresholds-for-main-mycorrhizal-type}}

\hypertarget{family-classification-based-on-different-thresholds}{%
\subsection{Family classification based on different
thresholds}\label{family-classification-based-on-different-thresholds}}

\begin{longtable}{l|l|l|l|l|l}
\caption{\label{tab:unnamed-chunk-5}Mycorrhizal type assigned to each family based on 4 different percentage thresholds (50, 60, 80 and 100)}\\
\hline
  & Family & 0.5 & 0.6 & 0.8 & 1\\
\hline
1 & Acanthaceae & AM & AM & MIX & MIX\\
\hline
2 & Achariaceae & AM & AM & AM & AM\\
\hline
5 & Actinidiaceae & AM & AM & AM & AM\\
\hline
6 & Adoxaceae & AM & AM & AM & MIX\\
\hline
7 & Aextoxicaceae & AM & AM & AM & AM\\
\hline
8 & Aizoaceae & NM & MIX & MIX & MIX\\
\hline
9 & Akaniaceae & AM & AM & AM & AM\\
\hline
11 & Alseuosmiaceae & AM & AM & AM & AM\\
\hline
12 & Alstroemeriaceae & AM & AM & AM & AM\\
\hline
13 & Altingiaceae & AM & AM & AM & AM\\
\hline
14 & Amaranthaceae & NM & MIX & MIX & MIX\\
\hline
15 & Amaryllidaceae & AM & AM & AM & MIX\\
\hline
16 & Amborellaceae & AM & AM & AM & AM\\
\hline
18 & Anacardiaceae & AM & AM & AM & AM\\
\hline
20 & Ancistrocladaceae & AM & AM & AM & AM\\
\hline
21 & Anisophylleaceae & AM & AM & AM & AM\\
\hline
22 & Annonaceae & AM & AM & AM & AM\\
\hline
24 & Aphloiaceae & AM & AM & AM & AM\\
\hline
25 & Apiaceae & AM & AM & AM & MIX\\
\hline
26 & Apocynaceae & AM & AM & AM & MIX\\
\hline
27 & Apodanthaceae & NM & NM & NM & NM\\
\hline
28 & Aponogetonaceae & MIX & MIX & MIX & MIX\\
\hline
29 & Aquifoliaceae & AM & AM & AM & AM\\
\hline
30 & Araceae & AM & MIX & MIX & MIX\\
\hline
31 & Araliaceae & AM & AM & AM & MIX\\
\hline
32 & Araucariaceae & AM & AM & AM & AM\\
\hline
33 & Arecaceae & AM & AM & AM & MIX\\
\hline
34 & Argophyllaceae & AM & AM & AM & AM\\
\hline
35 & Aristolochiaceae & AM & AM & MIX & MIX\\
\hline
36 & Asparagaceae & AM & AM & AM & MIX\\
\hline
37 & Asteliaceae & AM & AM & AM & AM\\
\hline
39 & Atherospermataceae & AM & AM & AM & AM\\
\hline
40 & Austrobaileyaceae & AM & AM & AM & AM\\
\hline
41 & Balanopaceae & AM & AM & AM & AM\\
\hline
42 & Balanophoraceae & NM & NM & NM & NM\\
\hline
43 & Balsaminaceae & AM & AM & AM & AM\\
\hline
45 & Barbeyaceae & AM & AM & AM & AM\\
\hline
46 & Basellaceae & AM & AM & AM & AM\\
\hline
48 & Begoniaceae & AM & AM & AM & AM\\
\hline
49 & Berberidaceae & AM & AM & AM & AM\\
\hline
51 & Betulaceae & EM & EM & EM & EM\\
\hline
52 & Biebersteiniaceae & AM & AM & AM & AM\\
\hline
53 & Bignoniaceae & AM & AM & AM & MIX\\
\hline
54 & Bixaceae & AM & AM & AM & AM\\
\hline
55 & Blandfordiaceae & AM & AM & AM & AM\\
\hline
56 & Bonnetiaceae & AM & AM & AM & AM\\
\hline
57 & Boraginaceae & AM & AM & MIX & MIX\\
\hline
58 & Boryaceae & AM & AM & AM & AM\\
\hline
59 & Brassicaceae & NM & NM & NM & NM\\
\hline
60 & Bromeliaceae & NM & NM & MIX & MIX\\
\hline
61 & Brunelliaceae & AM & AM & AM & AM\\
\hline
62 & Bruniaceae & AM & AM & AM & AM\\
\hline
63 & Burmanniaceae & AM & AM & AM & AM\\
\hline
64 & Burseraceae & AM & AM & AM & MIX\\
\hline
65 & Butomaceae & NM & NM & NM & NM\\
\hline
66 & Buxaceae & AM & AM & AM & AM\\
\hline
67 & Byblidaceae & NM & NM & NM & NM\\
\hline
68 & Cabombaceae & NM & NM & NM & NM\\
\hline
69 & Cactaceae & AM & AM & AM & MIX\\
\hline
70 & Calceolariaceae & AM & AM & AM & AM\\
\hline
71 & Calophyllaceae & AM & AM & AM & AM\\
\hline
72 & Calycanthaceae & AM & AM & AM & AM\\
\hline
73 & Calyceraceae & AM & AM & AM & AM\\
\hline
74 & Campanulaceae & AM & AM & AM & MIX\\
\hline
75 & Campynemataceae & AM & AM & AM & AM\\
\hline
76 & Canellaceae & AM & AM & AM & AM\\
\hline
77 & Cannabaceae & AM & AM & AM & AM\\
\hline
78 & Cannaceae & AM & AM & AM & AM\\
\hline
79 & Capparaceae & AM & AM & MIX & MIX\\
\hline
80 & Caprifoliaceae & AM & AM & AM & AM\\
\hline
81 & Cardiopteridaceae & AM & AM & AM & AM\\
\hline
82 & Caricaceae & AM & AM & AM & AM\\
\hline
83 & Carlemanniaceae & AM & AM & AM & AM\\
\hline
84 & Caryocaraceae & AM & AM & AM & AM\\
\hline
85 & Caryophyllaceae & NM & NM & MIX & MIX\\
\hline
86 & Casuarinaceae & AM & MIX & MIX & MIX\\
\hline
87 & Celastraceae & AM & AM & AM & MIX\\
\hline
89 & Cephalotaceae & NM & NM & NM & NM\\
\hline
90 & Ceratophyllaceae & NM & NM & NM & NM\\
\hline
91 & Cercidiphyllaceae & AM & AM & AM & AM\\
\hline
92 & Chloranthaceae & AM & AM & AM & AM\\
\hline
93 & Chrysobalanaceae & AM & AM & AM & AM\\
\hline
94 & Circaeasteraceae & AM & AM & AM & AM\\
\hline
95 & Cistaceae & EM & EM & EM & EM\\
\hline
96 & Cleomaceae & AM & MIX & MIX & MIX\\
\hline
97 & Clethraceae & AM & AM & AM & AM\\
\hline
98 & Clusiaceae & AM & AM & AM & AM\\
\hline
99 & Colchicaceae & AM & AM & AM & AM\\
\hline
100 & Columelliaceae & AM & AM & AM & AM\\
\hline
101 & Combretaceae & AM & AM & AM & AM\\
\hline
102 & Commelinaceae & NM & MIX & MIX & MIX\\
\hline
103 & Asteraceae & AM & AM & AM & MIX\\
\hline
104 & Connaraceae & AM & AM & AM & AM\\
\hline
105 & Convolvulaceae & AM & AM & MIX & MIX\\
\hline
106 & Coriariaceae & AM & AM & AM & AM\\
\hline
107 & Cornaceae & AM & AM & AM & AM\\
\hline
108 & Corsiaceae & AM & AM & AM & AM\\
\hline
109 & Corynocarpaceae & AM & AM & AM & AM\\
\hline
110 & Costaceae & AM & AM & AM & AM\\
\hline
111 & Crassulaceae & NM & NM & MIX & MIX\\
\hline
113 & Ctenolophonaceae & AM & AM & AM & AM\\
\hline
114 & Cucurbitaceae & AM & AM & AM & AM\\
\hline
115 & Cunoniaceae & AM & AM & AM & MIX\\
\hline
116 & Cupressaceae & AM & AM & AM & AM\\
\hline
117 & Curtisiaceae & AM & AM & AM & AM\\
\hline
118 & Cycadaceae & AM & AM & AM & AM\\
\hline
119 & Cyclanthaceae & NM & NM & MIX & MIX\\
\hline
120 & Cymodoceaceae & NM & NM & NM & NM\\
\hline
121 & Cynomoriaceae & NM & NM & NM & NM\\
\hline
122 & Cyperaceae & NM & MIX & MIX & MIX\\
\hline
123 & Cyrillaceae & AM & AM & AM & AM\\
\hline
124 & Cytinaceae & AM & AM & AM & AM\\
\hline
125 & Daphniphyllaceae & AM & AM & AM & AM\\
\hline
127 & Datiscaceae & AM & AM & AM & AM\\
\hline
128 & Degeneriaceae & AM & AM & AM & AM\\
\hline
129 & Diapensiaceae & ER & ER & ER & ER\\
\hline
130 & Dichapetalaceae & AM & AM & AM & AM\\
\hline
132 & Dilleniaceae & AM & AM & AM & AM\\
\hline
134 & Dioscoreaceae & AM & AM & AM & AM\\
\hline
135 & Dipentodontaceae & AM & AM & AM & AM\\
\hline
136 & Dipterocarpaceae & EM & EM & EM & EM\\
\hline
137 & Dirachmaceae & AM & AM & AM & AM\\
\hline
138 & Doryanthaceae & AM & AM & AM & AM\\
\hline
139 & Droseraceae & NM & NM & NM & NM\\
\hline
140 & Drosophyllaceae & NM & NM & NM & NM\\
\hline
141 & Ebenaceae & AM & AM & AM & AM\\
\hline
142 & Ecdeiocoleaceae & AM & AM & AM & AM\\
\hline
143 & Elaeagnaceae & AM & AM & AM & AM\\
\hline
144 & Elaeocarpaceae & AM & AM & AM & AM\\
\hline
145 & Elatinaceae & NM & NM & MIX & MIX\\
\hline
146 & Emblingiaceae & NM & NM & NM & NM\\
\hline
147 & Ephedraceae & AM & AM & AM & AM\\
\hline
149 & Eriocaulaceae & AM & AM & AM & MIX\\
\hline
150 & Erythroxylaceae & AM & AM & AM & AM\\
\hline
151 & Escalloniaceae & AM & AM & AM & AM\\
\hline
152 & Eucommiaceae & AM & AM & AM & AM\\
\hline
153 & Euphorbiaceae & AM & AM & AM & MIX\\
\hline
154 & Euphroniaceae & AM & AM & AM & AM\\
\hline
155 & Eupomatiaceae & AM & AM & AM & AM\\
\hline
156 & Eupteleaceae & AM & AM & AM & AM\\
\hline
157 & Fagaceae & EM & EM & EM & EM\\
\hline
158 & Flagellariaceae & AM & AM & AM & AM\\
\hline
159 & Fouquieriaceae & AM & AM & AM & AM\\
\hline
160 & Frankeniaceae & NM & NM & NM & NM\\
\hline
161 & Garryaceae & AM & AM & AM & AM\\
\hline
163 & Gelsemiaceae & AM & AM & AM & AM\\
\hline
164 & Gentianaceae & AM & AM & AM & MIX\\
\hline
165 & Geraniaceae & AM & AM & AM & AM\\
\hline
167 & Gesneriaceae & AM & AM & AM & AM\\
\hline
168 & Ginkgoaceae & AM & AM & AM & AM\\
\hline
170 & Gnetaceae & EM & EM & EM & EM\\
\hline
171 & Gomortegaceae & AM & AM & AM & AM\\
\hline
172 & Goodeniaceae & AM & MIX & MIX & MIX\\
\hline
173 & Goupiaceae & AM & AM & AM & AM\\
\hline
174 & Grossulariaceae & AM & AM & AM & AM\\
\hline
175 & Grubbiaceae & AM & AM & AM & AM\\
\hline
176 & Gunneraceae & AM & AM & AM & AM\\
\hline
177 & Gyrostemonaceae & NM & NM & NM & NM\\
\hline
178 & Haemodoraceae & NM & NM & NM & NM\\
\hline
179 & Halophytaceae & NM & NM & NM & NM\\
\hline
180 & Haloragaceae & AM & AM & MIX & MIX\\
\hline
181 & Hamamelidaceae & AM & AM & AM & AM\\
\hline
182 & Hanguanaceae & AM & AM & AM & AM\\
\hline
183 & Haptanthaceae & AM & AM & AM & AM\\
\hline
184 & Heliconiaceae & AM & AM & AM & AM\\
\hline
185 & Helwingiaceae & AM & AM & AM & AM\\
\hline
186 & Hernandiaceae & AM & AM & AM & AM\\
\hline
187 & Himantandraceae & AM & AM & AM & AM\\
\hline
188 & Huaceae & AM & AM & AM & AM\\
\hline
189 & Humiriaceae & AM & AM & AM & AM\\
\hline
191 & Hydnoraceae & AM & AM & AM & AM\\
\hline
192 & Hydrangeaceae & AM & AM & AM & MIX\\
\hline
193 & Hydrocharitaceae & NM & MIX & MIX & MIX\\
\hline
196 & Hypericaceae & AM & AM & AM & AM\\
\hline
197 & Hypoxidaceae & AM & AM & AM & AM\\
\hline
198 & Icacinaceae & AM & AM & AM & AM\\
\hline
199 & Iridaceae & AM & AM & AM & MIX\\
\hline
200 & Irvingiaceae & AM & AM & AM & AM\\
\hline
201 & Iteaceae & AM & AM & AM & AM\\
\hline
202 & Ixioliriaceae & AM & AM & AM & AM\\
\hline
203 & Ixonanthaceae & AM & AM & AM & AM\\
\hline
204 & Joinvilleaceae & AM & AM & AM & AM\\
\hline
205 & Juglandaceae & EM & MIX & MIX & MIX\\
\hline
207 & Juncaginaceae & NM & NM & NM & MIX\\
\hline
208 & Kirkiaceae & AM & AM & AM & AM\\
\hline
209 & Koeberliniaceae & NM & NM & NM & NM\\
\hline
210 & Krameriaceae & AM & AM & AM & AM\\
\hline
211 & Lacistemataceae & AM & AM & AM & AM\\
\hline
212 & Lactoridaceae & AM & AM & AM & AM\\
\hline
213 & Lamiaceae & AM & AM & AM & MIX\\
\hline
214 & Lanariaceae & AM & AM & AM & AM\\
\hline
215 & Lardizabalaceae & AM & AM & AM & AM\\
\hline
216 & Lauraceae & AM & AM & AM & MIX\\
\hline
217 & Lecythidaceae & AM & AM & AM & MIX\\
\hline
218 & Fabaceae & AM & AM & AM & MIX\\
\hline
219 & Lentibulariaceae & NM & NM & MIX & MIX\\
\hline
220 & Lepidobotryaceae & AM & AM & AM & AM\\
\hline
221 & Liliaceae & AM & AM & AM & AM\\
\hline
223 & Limnanthaceae & NM & NM & NM & NM\\
\hline
224 & Linaceae & AM & AM & AM & AM\\
\hline
225 & Linderniaceae & NM & MIX & MIX & MIX\\
\hline
226 & Loasaceae & NM & NM & NM & MIX\\
\hline
227 & Loganiaceae & AM & AM & AM & AM\\
\hline
228 & Loranthaceae & NM & NM & NM & NM\\
\hline
229 & Lowiaceae & AM & AM & AM & AM\\
\hline
230 & Lythraceae & AM & AM & AM & MIX\\
\hline
231 & Magnoliaceae & AM & AM & AM & AM\\
\hline
232 & Malpighiaceae & AM & AM & AM & AM\\
\hline
233 & Malvaceae & AM & AM & AM & MIX\\
\hline
234 & Marantaceae & AM & AM & MIX & MIX\\
\hline
235 & Marcgraviaceae & AM & AM & AM & AM\\
\hline
236 & Martyniaceae & AM & AM & AM & AM\\
\hline
238 & Melanthiaceae & AM & AM & AM & AM\\
\hline
239 & Melastomataceae & AM & AM & AM & AM\\
\hline
240 & Meliaceae & AM & AM & AM & AM\\
\hline
241 & Melianthaceae & AM & AM & AM & AM\\
\hline
242 & Menispermaceae & AM & AM & AM & AM\\
\hline
243 & Menyanthaceae & NM & MIX & MIX & MIX\\
\hline
245 & Misodendraceae & NM & NM & NM & NM\\
\hline
246 & Mitrastemonaceae & NM & NM & NM & NM\\
\hline
247 & Molluginaceae & NM & NM & MIX & MIX\\
\hline
248 & Monimiaceae & AM & AM & AM & AM\\
\hline
249 & Montiaceae & AM & AM & MIX & MIX\\
\hline
250 & Montiniaceae & AM & AM & AM & AM\\
\hline
251 & Moraceae & AM & AM & AM & MIX\\
\hline
252 & Moringaceae & AM & AM & AM & AM\\
\hline
253 & Muntingiaceae & AM & AM & AM & AM\\
\hline
254 & Musaceae & AM & AM & AM & AM\\
\hline
255 & Myodocarpaceae & AM & AM & AM & AM\\
\hline
256 & Myricaceae & AM & AM & MIX & MIX\\
\hline
257 & Myristicaceae & AM & AM & AM & AM\\
\hline
258 & Myrothamnaceae & AM & AM & AM & AM\\
\hline
259 & Myrtaceae & AM & AM & AM & MIX\\
\hline
260 & Nartheciaceae & AM & AM & AM & MIX\\
\hline
261 & Nelumbonaceae & NM & NM & NM & NM\\
\hline
263 & Neuradaceae & AM & AM & AM & AM\\
\hline
264 & Nitrariaceae & AM & AM & AM & AM\\
\hline
265 & Nothofagaceae & EM & EM & EM & EM\\
\hline
266 & Nyctaginaceae & MIX & MIX & MIX & MIX\\
\hline
267 & Nymphaeaceae & NM & NM & MIX & MIX\\
\hline
268 & Ochnaceae & AM & AM & MIX & MIX\\
\hline
269 & Olacaceae & AM & AM & AM & MIX\\
\hline
270 & Oleaceae & AM & AM & AM & AM\\
\hline
271 & Onagraceae & AM & AM & AM & MIX\\
\hline
272 & Oncothecaceae & AM & AM & AM & AM\\
\hline
275 & Orobanchaceae & NM & NM & MIX & MIX\\
\hline
276 & Oxalidaceae & AM & AM & AM & AM\\
\hline
277 & Paeoniaceae & AM & AM & AM & AM\\
\hline
278 & Pandaceae & AM & AM & AM & AM\\
\hline
279 & Pandanaceae & AM & AM & MIX & MIX\\
\hline
280 & Papaveraceae & NM & MIX & MIX & MIX\\
\hline
281 & Paracryphiaceae & AM & AM & AM & AM\\
\hline
282 & Passifloraceae & AM & AM & AM & AM\\
\hline
283 & Paulowniaceae & AM & AM & MIX & MIX\\
\hline
284 & Pedaliaceae & AM & AM & MIX & MIX\\
\hline
285 & Penaeaceae & AM & AM & AM & AM\\
\hline
286 & Pentadiplandraceae & NM & NM & NM & NM\\
\hline
287 & Pentaphragmataceae & AM & AM & AM & AM\\
\hline
288 & Pentaphylacaceae & AM & AM & AM & AM\\
\hline
290 & Peridiscaceae & AM & AM & AM & AM\\
\hline
291 & Petermanniaceae & AM & AM & AM & AM\\
\hline
292 & Petrosaviaceae & AM & AM & AM & AM\\
\hline
293 & Philesiaceae & AM & AM & AM & AM\\
\hline
294 & Philydraceae & AM & AM & AM & AM\\
\hline
295 & Phrymaceae & AM & AM & AM & AM\\
\hline
296 & Phyllanthaceae & AM & AM & AM & MIX\\
\hline
297 & Phyllonomaceae & AM & AM & AM & AM\\
\hline
299 & Phytolaccaceae & NM & NM & MIX & MIX\\
\hline
300 & Picramniaceae & AM & AM & AM & AM\\
\hline
301 & Picrodendraceae & AM & AM & AM & AM\\
\hline
302 & Pinaceae & EM & EM & EM & EM\\
\hline
303 & Piperaceae & AM & MIX & MIX & MIX\\
\hline
304 & Pittosporaceae & AM & AM & AM & AM\\
\hline
305 & Plantaginaceae & AM & AM & MIX & MIX\\
\hline
306 & Platanaceae & AM & AM & AM & AM\\
\hline
307 & Plocospermataceae & AM & AM & AM & AM\\
\hline
308 & Plumbaginaceae & NM & MIX & MIX & MIX\\
\hline
309 & Poaceae & AM & AM & AM & MIX\\
\hline
310 & Podocarpaceae & AM & AM & AM & AM\\
\hline
311 & Podostemaceae & NM & NM & NM & NM\\
\hline
312 & Polemoniaceae & AM & AM & AM & MIX\\
\hline
313 & Polygalaceae & AM & AM & AM & AM\\
\hline
314 & Polygonaceae & MIX & MIX & MIX & MIX\\
\hline
316 & Portulacaceae & NM & NM & MIX & MIX\\
\hline
317 & Posidoniaceae & NM & NM & NM & NM\\
\hline
318 & Potamogetonaceae & NM & NM & NM & MIX\\
\hline
319 & Primulaceae & AM & AM & AM & AM\\
\hline
320 & Proteaceae & NM & NM & NM & MIX\\
\hline
321 & Putranjivaceae & AM & AM & AM & AM\\
\hline
322 & Quillajaceae & AM & AM & AM & AM\\
\hline
323 & Rafflesiaceae & NM & NM & NM & NM\\
\hline
324 & Ranunculaceae & AM & AM & AM & MIX\\
\hline
325 & Rapateaceae & AM & AM & AM & AM\\
\hline
326 & Resedaceae & NM & NM & NM & NM\\
\hline
327 & Restionaceae & NM & MIX & MIX & MIX\\
\hline
329 & Rhamnaceae & AM & AM & AM & MIX\\
\hline
330 & Rhipogonaceae & AM & AM & AM & AM\\
\hline
331 & Rhizophoraceae & AM & MIX & MIX & MIX\\
\hline
332 & Roridulaceae & NM & NM & NM & NM\\
\hline
333 & Rosaceae & AM & AM & AM & MIX\\
\hline
334 & Rousseaceae & AM & AM & AM & AM\\
\hline
335 & Rubiaceae & AM & AM & AM & MIX\\
\hline
337 & Rutaceae & AM & AM & AM & MIX\\
\hline
338 & Sabiaceae & AM & AM & AM & AM\\
\hline
339 & Salicaceae & AM & AM & MIX & MIX\\
\hline
340 & Salvadoraceae & NM & NM & MIX & MIX\\
\hline
341 & Santalaceae & NM & MIX & MIX & MIX\\
\hline
342 & Sapindaceae & AM & AM & AM & AM\\
\hline
343 & Sapotaceae & AM & AM & AM & MIX\\
\hline
344 & Sarcobataceae & NM & NM & NM & NM\\
\hline
345 & Sarcolaenaceae & EM & EM & EM & EM\\
\hline
346 & Sarraceniaceae & NM & NM & NM & NM\\
\hline
347 & Saururaceae & AM & AM & AM & AM\\
\hline
348 & Saxifragaceae & NM & MIX & MIX & MIX\\
\hline
350 & Schisandraceae & AM & AM & AM & AM\\
\hline
351 & Schlegeliaceae & AM & AM & AM & AM\\
\hline
352 & Schoepfiaceae & AM & AM & AM & AM\\
\hline
353 & Sciadopityaceae & AM & AM & AM & AM\\
\hline
354 & Scrophulariaceae & AM & AM & AM & MIX\\
\hline
356 & Simaroubaceae & AM & AM & AM & AM\\
\hline
357 & Simmondsiaceae & AM & AM & AM & AM\\
\hline
358 & Siparunaceae & AM & AM & AM & AM\\
\hline
359 & Sladeniaceae & AM & AM & AM & AM\\
\hline
360 & Smilacaceae & AM & AM & AM & AM\\
\hline
361 & Solanaceae & AM & AM & AM & AM\\
\hline
362 & Sphaerosepalaceae & AM & AM & AM & AM\\
\hline
363 & Sphenocleaceae & NM & NM & NM & NM\\
\hline
364 & Stachyuraceae & AM & AM & AM & AM\\
\hline
365 & Staphyleaceae & AM & AM & AM & AM\\
\hline
366 & Stegnospermataceae & AM & AM & AM & AM\\
\hline
367 & Stemonaceae & AM & AM & AM & AM\\
\hline
368 & Stemonuraceae & AM & AM & AM & AM\\
\hline
369 & Stilbaceae & AM & AM & AM & AM\\
\hline
371 & Strelitziaceae & AM & AM & AM & AM\\
\hline
372 & Stylidiaceae & AM & AM & AM & AM\\
\hline
373 & Styracaceae & AM & AM & AM & AM\\
\hline
374 & Surianaceae & AM & AM & AM & AM\\
\hline
375 & Symplocaceae & AM & AM & AM & AM\\
\hline
378 & Tapisciaceae & AM & AM & AM & AM\\
\hline
379 & Taxaceae & AM & AM & AM & AM\\
\hline
380 & Tecophilaeaceae & AM & AM & AM & AM\\
\hline
381 & Tetrachondraceae & AM & AM & AM & AM\\
\hline
382 & Tetramelaceae & AM & AM & AM & AM\\
\hline
383 & Tetrameristaceae & AM & AM & AM & AM\\
\hline
384 & Theaceae & AM & AM & AM & AM\\
\hline
385 & Thurniaceae & AM & AM & AM & AM\\
\hline
386 & Thymelaeaceae & AM & AM & AM & MIX\\
\hline
387 & Ticodendraceae & EM & EM & EM & EM\\
\hline
388 & Tofieldiaceae & AM & AM & MIX & MIX\\
\hline
389 & Torricelliaceae & AM & AM & AM & AM\\
\hline
391 & Trigoniaceae & AM & AM & AM & AM\\
\hline
392 & Triuridaceae & AM & AM & AM & AM\\
\hline
393 & Trochodendraceae & AM & AM & AM & AM\\
\hline
394 & Tropaeolaceae & AM & AM & AM & AM\\
\hline
396 & Ulmaceae & AM & AM & AM & AM\\
\hline
397 & Urticaceae & AM & MIX & MIX & MIX\\
\hline
398 & Vahliaceae & AM & AM & AM & AM\\
\hline
399 & Velloziaceae & AM & AM & AM & AM\\
\hline
400 & Verbenaceae & AM & AM & AM & AM\\
\hline
401 & Violaceae & AM & AM & AM & AM\\
\hline
402 & Vitaceae & AM & AM & AM & AM\\
\hline
403 & Vivianiaceae & AM & AM & AM & AM\\
\hline
404 & Vochysiaceae & AM & AM & AM & AM\\
\hline
405 & Welwitschiaceae & AM & AM & AM & AM\\
\hline
406 & Winteraceae & AM & AM & AM & AM\\
\hline
407 & Xanthorrhoeaceae & AM & AM & AM & AM\\
\hline
408 & Xeronemataceae & AM & AM & AM & AM\\
\hline
410 & Zamiaceae & AM & AM & AM & AM\\
\hline
411 & Zingiberaceae & AM & AM & AM & AM\\
\hline
412 & Zosteraceae & NM & NM & NM & NM\\
\hline
413 & Zygophyllaceae & AM & AM & MIX & MIX\\
\hline
\end{longtable}

\hypertarget{phylogenetic-signal-of-diversification-rates-age-and-richness}{%
\section{Phylogenetic signal of diversification rates, age, and
richness}\label{phylogenetic-signal-of-diversification-rates-age-and-richness}}

\begin{table}[H]

\caption{\label{tab:unnamed-chunk-6}Phylogenetic signal of the four response variables}
\centering
\begin{tabular}{l|r}
\hline
Variable & Lambda\\
\hline
r (epsilon = 0) & 0.4681621\\
\hline
r (epsilon = 0.9) & 0.3510291\\
\hline
Stem Age & 1.0000000\\
\hline
Richness & 0.0000010\\
\hline
\end{tabular}
\end{table}

\hypertarget{phylogenetic-anova}{%
\section{Phylogenetic ANOVA}\label{phylogenetic-anova}}

\hypertarget{summary-statistics}{%
\subsection{Summary statistics}\label{summary-statistics}}

\begin{table}[H]

\caption{\label{tab:unnamed-chunk-7}phyANOVA summary statistics for both values of epsilon}
\centering
\begin{tabular}{r|r|r}
\hline
epsilon & F & pvalue\\
\hline
0.0 & 6.151844 & 0.017\\
\hline
0.9 & 5.974891 & 0.021\\
\hline
\end{tabular}
\end{table}

\hypertarget{posthoc-tests}{%
\subsection{Posthoc tests}\label{posthoc-tests}}

\begin{table}[H]

\caption{\label{tab:unnamed-chunk-8}Pairwise Corrected p-values for epsilon = 0}
\centering
\begin{tabular}{l|r|r|r|r}
\hline
  & AM & EM & MIX & NM\\
\hline
AM & 1.000 & 1 & 0.006 & 1.00\\
\hline
EM & 1.000 & 1 & 1.000 & 1.00\\
\hline
MIX & 0.006 & 1 & 1.000 & 0.01\\
\hline
NM & 1.000 & 1 & 0.010 & 1.00\\
\hline
\end{tabular}
\end{table}

\begin{table}[H]

\caption{\label{tab:unnamed-chunk-9}Pairwise Corrected p-values for epsilon = 0.9}
\centering
\begin{tabular}{l|r|r|r|r}
\hline
  & AM & EM & MIX & NM\\
\hline
AM & 1.00 & 1.000 & 0.010 & 1.000\\
\hline
EM & 1.00 & 1.000 & 0.976 & 1.000\\
\hline
MIX & 0.01 & 0.976 & 1.000 & 0.006\\
\hline
NM & 1.00 & 1.000 & 0.006 & 1.000\\
\hline
\end{tabular}
\end{table}

\hypertarget{species-level-database}{%
\section{Species-level database}\label{species-level-database}}

\hypertarget{clean-database---excluding-species-with-any-remark}{%
\subsection{Clean database - excluding species with any
remark}\label{clean-database---excluding-species-with-any-remark}}

\hypertarget{boxplots-1}{%
\subsubsection{Boxplots}\label{boxplots-1}}

\begin{figure}
\centering
\includegraphics{supp_mat_files/figure-latex/unnamed-chunk-11-1.pdf}
\caption{Relationship between mycorrhizal type and diversification
rates. a) diversification rate estimated with ε (relative extinction
fraction) = 0 and b) diversification rate estimated with ε= 0.9. AM:
Arbuscular mycorrhiza, EM: Ectomycorrhiza, NM: non-mycorrhizal and MIX
(families with no dominance of any specific mycorrhizal association).
The size of the points indicates the Mycorrhizal Type Diversity Index
value for each lineage, indicating a predominance of larger indices with
higher diversification rates.}
\end{figure}

\hypertarget{scatterplots-1}{%
\subsubsection{Scatterplots}\label{scatterplots-1}}

\begin{figure}
\centering
\includegraphics{supp_mat_files/figure-latex/unnamed-chunk-12-1.pdf}
\caption{Scatterplots showing the relationship between mycorrhizal
diversity index and diversification rates (a and c), species richness
(b) and age family (d). Diversification rates were estimated with ε
(relative extinction fraction) = 0 (a) and with ε= 0.9 (c). The red and
blue lines indicate the results of a linear model and a phylogenetic
generalized least squares (PGLS) fit, respectively.}
\end{figure}

\hypertarget{more-inclusive-database---including-species-with-any-remark}{%
\subsection{More inclusive database - including species with any
remark}\label{more-inclusive-database---including-species-with-any-remark}}

\hypertarget{boxplots-2}{%
\subsubsection{Boxplots}\label{boxplots-2}}

\begin{figure}
\centering
\includegraphics{supp_mat_files/figure-latex/unnamed-chunk-13-1.pdf}
\caption{Relationship between mycorrhizal type and diversification
rates. a) diversification rate estimated with ε (relative extinction
fraction) = 0 and b) diversification rate estimated with ε= 0.9. AM:
Arbuscular mycorrhiza, EM: Ectomycorrhiza, NM: non-mycorrhizal and MIX
(families with no dominance of any specific mycorrhizal association).
The size of the points indicates the Mycorrhizal Type Diversity Index
value for each lineage, indicating a predominance of larger indices with
higher diversification rates.}
\end{figure}

\hypertarget{scatterplots-2}{%
\subsubsection{Scatterplots}\label{scatterplots-2}}

\begin{figure}
\centering
\includegraphics{supp_mat_files/figure-latex/unnamed-chunk-14-1.pdf}
\caption{Scatterplots showing the relationship between mycorrhizal
diversity index and diversification rates (a and c), species richness
(b) and age family (d). Diversification rates were estimated with ε
(relative extinction fraction) = 0 (a) and with ε= 0.9 (c). The red and
blue lines indicate the results of a linear model and a phylogenetic
generalized least squares (PGLS) fit, respectively.}
\end{figure}


\end{document}
